\documentclass[../../../main]{subfiles}

\begin{document}
    \begin{table}[H]
        \centering
        \caption{ページング処理における計測結果}
        \label{tab:result-paging}
        \begin{tabular}{|p{4cm}|p{10cm}|}
            \hline
            \textbf{計測項目}                & \textbf{結果}                                                                                                                                                                                                                                     \\ \hline
            サーバー内処理時間                    & イベント数に応じて処理時間は指数関数的に増加した. ページサイズが小さいほど増加速度は大きく, イベント数30万件時のページサイズ別処理時間は, 10件で約1051秒, 100件で約114秒, 1000件で約18.5秒だった. ページング手法間の差は小さく, イベント数が増えると多少の差が見られた. ページサイズ10000件, イベント数100万件の時, 末尾によるページングは28.978秒, OFFSETによるページングは27.333秒, 並行化した末尾によるページングは27.153秒であった.  \\ \hline
            アプリケーションサーバーコンテナのCPU使用率      & ページサイズやページング手法, イベント数に関わらず, CPU使用率は70\%から80\%の間で一定だった.                                                                                                                                                                                             \\ \hline
            DBサーバーコンテナのCPU使用率            & ページング手法に関わらず, イベント数に比例してCPU使用率は増加し, ページサイズが小さいほど使用率は高かった. イベント数100万件時の使用率は65\%から75\%程度だった.                                                                                                                                                          \\ \hline
            アプリケーションサーバーコンテナのメモリ使用量      & ページサイズやページング手法に関わらず, イベント数に比例してメモリ使用量は増加し, 75万件で上限の1024MiBに達した. ストリーミングやチャンク処理と異なり, メモリ使用量は時折950MiB程度まで下がることもあった.                                                                                                                                    \\ \hline
            DBサーバーコンテナのメモリ使用量            & ページサイズやページング手法に関わらず, イベント数に比例してメモリ使用量は増加し, 100万件時点では419MiBから430MiB程度だった.                                                                                                                                                                           \\ \hline
            アプリケーションサーバーコンテナのブロック入出力     & ページサイズやページング手法に関わらず, 0から75万件のイベント数まではブロック入出力が0MBであったが, 75万件を超えると指数関数的に増加した.                                                                                                                                                                        \\ \hline
            DBサーバーコンテナのブロック入出力           & ページサイズやページング手法に関わらず, 書き込み量はほぼ0MBで一定だったが, 読み込み量はイベント数に比例して増加し, 100万件時点で13926.4MBとなった.                                                                                                                                                                \\ \hline
            ネットワーク入出力                    & ページサイズやページング手法に関わらず, アプリケーションサーバーコンテナの送信量とDBサーバーコンテナの受信量, 及びその逆の通信量はほぼ一致した. アプリケーションサーバーコンテナの受信量はイベント数0件時点で2400MBから始まり, 100万件時点で44748.8MB, 送信量は0件時点でほぼ0MBから始まり, 100万件時点で1740.8MBとなった.                                                                  \\ \hline
            アプリケーションサーバーコンテナのCPU制限変更時の影響 & CPU制限の変更による顕著な変化は確認されなかった.                                                                                                                                                                                                                       \\ \hline
            アプリケーションサーバーコンテナのメモリ制限変更時の影響 & メモリ制限の変更による顕著な変化は確認されなかった.                                                                                                                                                                                                                       \\ \hline
            DBサーバーコンテナのCPU制限変更時の影響       & CPU制限の変更による顕著な変化は確認されなかった.                                                                                                                                                                                                                       \\ \hline
            DBサーバーコンテナのメモリ制限変更時の影響       & メモリ制限の変更による顕著な変化は確認されなかった.                                                                                                                                                                                                                       \\ \hline
        \end{tabular}
    \end{table}
\end{document}

