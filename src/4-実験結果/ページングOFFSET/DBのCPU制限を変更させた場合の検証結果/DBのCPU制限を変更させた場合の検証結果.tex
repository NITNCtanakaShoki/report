\documentclass[../../../../main]{subfiles}

\begin{document}
    \subsubsection{DBサーバーのCPU制限を変更した場合の検証結果}\label{subsubsec:result-paging-offset-change-db-cpu}

    本セクションでは, Web APIサーバーのアプリケーションサーバーコンテナにCPU制限400\%, メモリ制限8192MBを設定し, DBサーバーのメモリ制限を1024MBに固定した上で, DBサーバーコンテナのCPU制限を200\%, 300\%, 400\%に変更した

    際の実験結果を報告する.

    \subsubsubsection{サーバー内処理時間}

    図\ref{fig:paging-offset-change-db-cpu-limit-server-time-app_4_8192-db_1024}では, DBサーバーのCPU制限を変更した際のサーバー内処理時間の変化を示している.

    \subfile{図:app-server-time/図:app-server-time}

    この結果から, DBサーバーのCPU制限の変更がサーバー内処理時間に顕著な影響を与えることは確認されなかった.

    \subsubsubsection{CPU使用率}

    図\ref{fig:paging-offset-change-db-cpu-limit-db-cpu-app_4_8192-db_1024}では, 変更されたCPU制限によるDBサーバーコンテナのCPU使用率の変動を示す.

    \subfile{図:db-CPU/図:db-CPU}

    ここでも, DBサーバーコンテナのCPU使用率において, CPU制限の変更が有意な影響を及ぼすことは確認されなかった. その他の計測事項に関しても, CPU制限の変更が影響を及ぼさないことが観察されたため, その結果はここでは省略する.

\end{document}