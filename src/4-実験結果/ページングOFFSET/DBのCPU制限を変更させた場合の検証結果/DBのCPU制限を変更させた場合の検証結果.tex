\documentclass[../../../../main]{subfiles}

\begin{document}
    \subsubsection{DBサーバーのCPU制限を変更させた場合の検証結果}\label{subsubsec:result-paging-offset-change-db-cpu}

    Web APIサーバーのアプリケーションサーバーコンテナのCPU制限を400\%、メモリ制限8192MBとし、DBサーバーのメモリ制限を1024MBとして、DBサーバーコンテナのCPU制限を200\%、300\%、400\%と変更させた場合の実験結果を示す。

    \subsubsubsection{サーバー内処理時間}

    図\ref{fig:paging-offset-change-db-cpu-limit-server-time-app_4_8192-db_1024}に、CPU制限を変化させた際のサーバー内処理時間を示す。

    \subfile{図:app-server-time/図:app-server-time}

    DBサーバーのCPU制限がサーバー内処理時間に与える影響は確認できなかった。

    \subsubsubsection{CPU使用率}

    図\ref{fig:paging-offset-change-db-cpu-limit-db-cpu-app_4_8192-db_1024}にCPU制限を変化させた際のDBサーバーコンテナのCPU使用率を示す。

    \subfile{図:db-CPU/図:db-CPU}

    DBサーバーのCPU制限がDBサーバーコンテナのCPU使用率に与える影響は確認できなかった。その他の計測事項においても、CPU制限を変化させても影響がなかったため結果は省略する。

\end{document}