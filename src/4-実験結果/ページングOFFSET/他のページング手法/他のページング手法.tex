\documentclass[../../../../main]{subfiles}

\begin{document}
    \subsubsection{他のページング手法との比較}\label{subsubsec:result-each-paging-only-limit}

    Web APIサーバーのアプリケーションサーバーコンテナをCPU制限100\%、メモリ制限1024MBとし、DBサーバーも同様の制限を設定した状態で、OFFSETを使用したページング手法、各ページの最後のイベントを使用したページング手法、各ページの最後のイベントを使用したページング手法の集計処理とDBアクセス処理を並行化したものを比較する。

    \subsubsubsection{イベント数とサーバー内処理時間の関係}\label{subsubsubsec:result-each-paging-only-limit-server-time}

    図\ref{fig:each-paging-server-time-app_1_1024-db_1_1024}に、イベント数に対するサーバー内処理時間の関係を示す。

    \subfile{図:detail-server-time/図:detail-server-time}

    図\ref{fig:each-paging-server-time-app_1_1024-db_1_1024}から、全体を通してほとんど差はないが、イベント数が増加するにつれて各ページの最後のイベントを使用したページング手法の方がサーバー内処理時間が比較的小さいことがわかる。またそれも並行化したほうがわずかに小さい。

    \subsubsubsection{イベント数とCPU使用率の関係}\label{subsubsubsec:result-each-paging-only-limit-cpu}

    図\ref{fig:each-paging-app-cpu-app_1_1024-db_1_1024}に、イベント数に対するアプリケーションサーバーコンテナのCPU使用率の関係を示す。

    \subfile{図:detail-app-cpu/図:detail-app-cpu}

    図\ref{fig:each-paging-app-cpu-app_1_1024-db_1_1024}から各ページング手法でCPU使用率が顕著に異なることはなく、およそ70\%から90\%ほどのCPU使用率であることがわかる。

    図\ref{fig:each-paging-db-cpu-app_1_1024-db_1_1024}に、イベント数に対するDBサーバーコンテナのCPU使用率の関係を示す。

    \subfile{図:detail-db-cpu/図:detail-db-cpu}

    図\ref{fig:each-paging-db-cpu-app_1_1024-db_1_1024}から、各ページング手法でDBのCPU使用率に変化はないことがわかった。

    \subsubsubsection{イベント数とメモリ使用量の関係}\label{subsubsubsec:result-each-paging-only-limit-mem}

    図\ref{fig:each-paging-app-mem-app_1_1024-db_1_1024}に、イベント数に対するアプリケーションサーバーコンテナのメモリ使用量の関係を示す。

    \subfile{図:detail-app-mem/図:detail-app-mem}

    図\ref{fig:each-paging-app-mem-app_1_1024-db_1_1024}からアプリケーションコンテナのメモリ使用量は各ページング手法で変化がないことがわかる。

    図\ref{fig:each-paging-db-mem-app_1_1024-db_1_1024}に、イベント数に対するDBサーバーコンテナのメモリ使用量の関係を示す。

    \subfile{図:detail-db-mem/図:detail-db-mem}

    図\ref{fig:each-paging-db-mem-app_1_1024-db_1_1024}より、DBも、各ページング手法でメモリ使用量に変化はないことがわかる。

    \subsubsubsection{イベント数とディスク書き込みの関係}\label{subsubsubsec:result-each-paging-only-limit-disk-in}

    図\ref{fig:each-paging-app-disk-in-app_1_1024-db_1_1024}に、イベント数に対するアプリケーションサーバーコンテナのディスク書き込み量の関係を示す。

    \subfile{図:detail-app-disk-in/図:detail-app-disk-in}

    図\ref{fig:each-paging-app-disk-in-app_1_1024-db_1_1024}から、各ページング手法でディスク書き込み量に変化はなく、全て仮想メモリを使用していることがわかる。

\end{document}