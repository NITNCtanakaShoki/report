\documentclass[../../../../main]{subfiles}

\begin{document}
    \subsubsection{アプリケーションサーバーのCPU制限を変更した場合の検証結果}\label{subsubsec:result-paging-offset-change-app-cpu}

    本セクションでは、Web APIサーバーのアプリケーションサーバーコンテナのメモリ制限を1024MBとし、DBサーバーのCPU制限を100\%、メモリ制限を1024MBに固定した上で、アプリケーションサーバーのCPU制限を200\%、300\%、400\%に変更した際の実験結果を報告する。

    \subsubsubsection{サーバー内処理時間}

    図\ref{fig:paging-offset-change-app-cpu-limit-server-time-app_1024-db_1_1024}では、CPU制限を変更した際のサーバー内処理時間を示している。

    \subfile{図:app-server-time/図:app-server-time}

    結果として、アプリケーションコンテナのCPU制限の変更がサーバー内処理時間に顕著な影響を与えることは確認されなかった。ただし、イベント数が7万から8万の範囲においては、他の範囲と比べて処理時間がわずかに増加する傾向が見られた。

    \subsubsubsection{CPU使用率}

    図\ref{fig:paging-offset-change-app-cpu-limit-app-cpu-app_1024-db_1_1024}にて、CPU制限の変更に伴うアプリケーションサーバーコンテナのCPU使用率を示す。

    \subfile{図:app-CPU/図:app-CPU}

    アプリケーションコンテナのCPU使用率については、使用可能なCPUの量を増加させても、その影響は見られなかった。その他の計測項目に関しても、CPU制限の変更が影響を及ぼさないことが確認されたため、その結果はここでは省略する。

\end{document}
