\documentclass[../../../../main]{subfiles}

\begin{document}
    \subsubsection{アプリケーションサーバーのCPU制限を変更させた場合の検証結果}\label{subsubsec:result-paging-offset-change-app-cpu}

    ここでは、Web APIサーバーのアプリケーションサーバーコンテナメモリ制限1024MBとし、DBサーバーのCPU制限を100\%、メモリ制限を1024MBとして、アプリケーションサーバーのCPU制限を200\%、300\%、400\%と変更させた場合の実験結果を示す。

    \subsubsubsection{サーバー内処理時間}

    図\ref{fig:paging-offset-change-app-cpu-limit-server-time-app_1024-db_1_1024}に、CPU制限を変化させた際のサーバー内処理時間を示す。

    \subfile{図:app-server-time/図:app-server-time}

    アプリケーションコンテナのCPU制限がサーバー内処理時間に与える影響は確認できなかった。しかし、どの計測でもイベント数が7万から8万の間だけ周囲に比べてサーバー内処理時間が少し増加していることがわかった。

    \subsubsubsection{CPU使用率}

    図\ref{fig:paging-offset-change-app-cpu-limit-app-cpu-app_1024-db_1_1024}にCPU制限を変化させた際のアプリケーションサーバーコンテナのCPU使用率を示す。

    \subfile{図:app-CPU/図:app-CPU}

    アプリケーションコンテナのCPU使用率は、使用可能なCPU量を増加しても影響がないことがわかった。その他の計測事項においても、CPU制限を変化させても影響がなかったため結果は省略する。



\end{document}