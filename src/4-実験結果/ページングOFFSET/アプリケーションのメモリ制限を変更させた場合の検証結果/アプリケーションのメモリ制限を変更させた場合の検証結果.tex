\documentclass[../../../../main]{subfiles}

\begin{document}
    \subsubsection{アプリケーションサーバーのメモリ制限を変更させた場合の検証結果}\label{subsubsec:result-paging-offset-change-app-memory}

    ここでは、Web APIサーバーのアプリケーションサーバーコンテナメモリ制限1024MBとし、DBサーバーのCPU制限を100\%、メモリ制限を1024MBとして、アプリケーションサーバーのCPU制限を200\%、300\%、400\%と変更させた場合の実験結果を示す。

    \subsubsubsection{サーバー内処理時間}

    図\ref{fig:paging-offset-change-app-memory-limit-server-time-app_4_db_1_1024}に、CPU制限を変化させた際のサーバー内処理時間を示す。

    \subfile{図:app-server-time/図:app-server-time}

    図\ref{fig:paging-offset-change-app-memory-limit-server-time-app_4_db_1_1024}より、仮想メモリを使用していないと推測されている75万ほどのイベント数ではほとんどCPU制限が100\%のときと変化がないが、仮想メモリを使い出したと推測される75万のイベント数以降は、サーバー内処理時間が増減していることがわかる。しかし、制限されるCPU量には関係が見られないため、ディスクの状態によって仮想メモリの速度がかわっているのではないかと推測される。

    \subsubsubsection{メモリ使用量}

    図\ref{fig:paging-offset-change-app-memory-limit-app-memory-app_4_db_1_1024}に、CPU制限を変化させた際のサーバー内処理時間を示す。

    \subfile{図:app-memory/図:app-memory}

    \subsubsubsection{アプリケーションのディスク使用量}

    \subfile{図:app-disk-in/図:app-disk-in}




\end{document}