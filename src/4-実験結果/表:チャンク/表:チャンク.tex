\documentclass[../../../main]{subfiles}

\begin{document}
    \begin{table}[H]
        \centering
        \caption{チャンク処理における計測結果}
        \label{tab:result-chunk}
        \begin{tabular}{|p{4cm}|p{10cm}|}
            \hline
            \textbf{計測項目}                & \textbf{結果}                                                                                                                                                                \\ \hline
            サーバー内処理時間                    & チャンクの大きさに関わらず, ユーザーイベント件数が0から75万件まで増加すると, 処理時間も比例して増加する傾向が見られた. 75万件を超えると増加率が顕著になり, 測定値は不均一となった. 75万件の時点での処理時間は約10.1855秒であった.                                                     \\ \hline
            アプリケーションサーバーコンテナのCPU使用率      & チャンクの大きさにかかわらず, イベント件数が0から75万件までの間, CPU使用率はほぼ100\%を維持した. 75万件を超えると, 使用率は84\%から96\%に低下した.                                                                                         \\ \hline
            DBサーバーコンテナのCPU使用率            & チャンクの大きさに関係なく, イベント件数の増加に伴い, CPU使用率は19\%から徐々に増加し, 最大90\%程度に達した. 使用率の増加に明確な規則性は見られなかった.                                                                                           \\ \hline
            アプリケーションサーバーコンテナのメモリ使用量      & チャンクの大きさにかかわらず, イベント件数の増加に比例してメモリ使用量は増加し, 100万件時点で432MiBになった.                                                                                                                  \\ \hline
            DBサーバーコンテナのメモリ使用量            & チャンクの大きさに関わらず, イベント件数に比例してメモリ使用量は増加し, 100万件時点で422.9MiBであった.                                                                                                                    \\ \hline
            アプリケーションサーバーコンテナのブロック入出力     & チャンクの大きさにかかわらず, 0から75万件のイベント件数ではブロック入出力が0MBであったが, 75万件を超えると指数関数的に増加し, 100万件時点で276480MBに達した.                                                                                     \\ \hline
            DBサーバーコンテナのブロック入出力           & チャンクの大きさにかかわらず, 書き込み量はほぼ0MBで一定であったが, 読み込み量はイベント件数に比例して増加し, 100万件時点で13926.4MBとなった.                                                                                               \\ \hline
            ネットワーク入出力                    & チャンクの大きさに関わらず, アプリケーションサーバーコンテナの送信量とDBサーバーコンテナの受信量, またその逆の通信量はほぼ一致した. アプリケーションサーバーコンテナの受信量はイベント件数0件時点で2400MBから増加し, 100万件時点で44544MBとなり, 送信量は0件時点でほぼ0MBから始まり, 100万件時点で1740.8MBとなった.  \\ \hline
            アプリケーションサーバーコンテナのCPU制限変更時の影響 & CPU制限の変更による顕著な変化は確認されなかった.                                                                                                                                                  \\ \hline
            アプリケーションサーバーコンテナのメモリ制限変更時の影響 & メモリ制限を4096MB以上に設定すると, サーバー内処理時間はイベント件数100万件まで比例して増加し, ブロック入出力量は全イベント件数で0となった.                                                                                                   \\ \hline
            DBサーバーコンテナのCPU制限変更時の影響       & CPU制限の変更による顕著な変化は確認されなかった.                                                                                                                                                  \\ \hline
            DBサーバーコンテナのメモリ制限変更時の影響       & メモリ制限の変更による顕著な変化は確認されなかった.                                                                                                                                                  \\ \hline
        \end{tabular}
    \end{table}
\end{document}
