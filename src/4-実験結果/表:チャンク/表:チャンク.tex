\documentclass[../../../main]{subfiles}

\begin{document}
    \begin{table}[H]
        \centering
        \caption{チャンク処理における計測結果}
        \label{tab:result-chunk}
        \begin{tabular}{|p{4cm}|p{10cm}|}
            \hline
            \textbf{計測項目}                & \textbf{結果}                                                                                                                                                                \\ \hline
            サーバー内処理時間                    & チャンクの大きさに関わらず、ユーザーイベント数が0から75万件まで増加すると、処理時間も比例して増加する傾向が見られた。75万件を超えると増加率が顕著になり、測定値は不均一となった。75万件の時点での処理時間は約10.1855秒であった。                                                    \\ \hline
            アプリケーションサーバーコンテナのCPU使用率      & チャンクの大きさにかかわらず、イベント数が0から75万件までの間、CPU使用率はほぼ100\%を維持した。75万件を超えると、使用率は84\%から96\%に低下した。                                                                                        \\ \hline
            DBサーバーコンテナのCPU使用率            & チャンクの大きさに関係なく、イベント数の増加に伴い、CPU使用率は19\%から徐々に増加し、最大90\%程度に達した。使用率の増加に明確な規則性は見られなかった。                                                                                          \\ \hline
            アプリケーションサーバーコンテナのメモリ使用量      & チャンクの大きさにかかわらず、イベント数の増加に比例してメモリ使用量は増加し、100万件時点で432MiBになった。                                                                                                                 \\ \hline
            DBサーバーコンテナのメモリ使用量            & チャンクの大きさに関わらず、イベント数に比例してメモリ使用量は増加し、100万件時点で422.9MiBであった。                                                                                                                   \\ \hline
            アプリケーションサーバーコンテナのブロック入出力     & チャンクの大きさにかかわらず、0から75万件のイベント数ではブロック入出力が0MBであったが、75万件を超えると指数関数的に増加し、100万件時点で276480MBに達した。                                                                                    \\ \hline
            DBサーバーコンテナのブロック入出力           & チャンクの大きさにかかわらず、書き込み量はほぼ0MBで一定であったが、読み込み量はイベント数に比例して増加し、100万件時点で13926.4MBとなった。                                                                                              \\ \hline
            ネットワーク入出力                    & チャンクの大きさに関わらず、アプリケーションサーバーコンテナの送信量とDBサーバーコンテナの受信量、またその逆の通信量はほぼ一致した。アプリケーションサーバーコンテナの受信量はイベント数0件時点で2400MBから増加し、100万件時点で44544MBとなり、送信量は0件時点でほぼ0MBから始まり、100万件時点で1740.8MBとなった。 \\ \hline
            アプリケーションサーバーコンテナのCPU制限変更時の影響 & CPU制限の変更による顕著な変化は確認されなかった。                                                                                                                                                 \\ \hline
            アプリケーションサーバーコンテナのメモリ制限変更時の影響 & メモリ制限を4096MB以上に設定すると、サーバー内処理時間はイベント数100万件まで比例して増加し、ブロック入出力量は全イベント数で0となった。                                                                                                  \\ \hline
            DBサーバーコンテナのCPU制限変更時の影響       & CPU制限の変更による顕著な変化は確認されなかった。                                                                                                                                                 \\ \hline
            DBサーバーコンテナのメモリ制限変更時の影響       & メモリ制限の変更による顕著な変化は確認されなかった。                                                                                                                                                 \\ \hline
        \end{tabular}
    \end{table}
\end{document}
