\documentclass[../../main]{subfiles}

\begin{document}
    \section{実験結果}\label{sec:result}

    \subsection{ストリーミング処理}\label{subsec:result-streaming}

    本節では、ストリーミング処理を用いたイベントソーシングの計測結果について述べる。以下の表\ref{tab:result-streaming}に、ストリーミング処理の計測結果を示す。測定条件は、アプリケーションサーバーコンテナとDBサーバーコンテナのCPU制限を100\%、メモリ制限を1024MBに設定した状態である。CPU制限の変更時の計測は、100\%、200\%、300\%、400\%で行ない、メモリ制限の変更時は、1024MB、2048MB、3072MB、4096MBで計測を行った。

    \subfile{表:ストリーミング/表:ストリーミング}

    \subsection{チャンク処理}\label{subsec:result-chunk}

    本節では、チャンク処理によるイベントソーシングの計測結果について述べる。以下の表\ref{tab:result-chunk}に、チャンク処理の計測結果を示す。基本的な測定条件はストリーミング処理と同様である。チャンクのサイズは10、100、1000、10000件で計測を行なった。CPU制限とメモリ制限の変更における計測条件もストリーミング処理と同じである。

    \subfile{表:チャンク/表:チャンク}

    \subsection{ページング処理}\label{subsec:result-paging}

    本節では、ページング処理を用いたイベントソーシングの計測結果について述べる。以下の表\ref{tab:result-paging}に、ページング処理の計測結果を示す。測定条件はストリーミング処理、チャンク処理と同様である。ページング処理では、ページサイズを10、100、1000、10000件で計測した。なお、ページサイズが10、100、1000件の場合はレスポンスタイムが長くなるため、イベント件数30万件までの計測に限定した。CPU制限とメモリ制限の変更時の計測条件も同様である。

    \subfile{表:ページング/表:ページング}

    \subsection{ストリーミング処理、チャンク処理、ページング処理の比較}\label{subsec:result-comparison}

    本節では、ストリーミング処理、チャンク処理、ページング処理の各手法を比較した結果について述べる。比較に用いた計測値は、アプリケーションサーバーコンテナとDBサーバーコンテナのCPU制限を100\%、メモリ制限を1024MBに設定した状態のものである。ただし、最大イベント数の計測は、各集計方法においてアプリケーションサーバーコンテナのCPU制限を400\%、メモリ制限を8192MB、DBサーバーコンテナのCPU制限を100\%、メモリ制限を1024MBに設定して行った。以下の表\ref{tab:result-comparison}に結果を示す。

    \subfile{表:各手法の比較/表:各手法の比較}

    \clearpage

\end{document}