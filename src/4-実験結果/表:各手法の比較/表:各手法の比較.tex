\documentclass[../../../main]{subfiles}

\begin{document}
    \begin{table}[H]
        \centering
        \caption{各集計方法における比較結果}
        \label{tab:result-comparison}
        \begin{tabular}{|p{4cm}|p{10cm}|}
            \hline
            \textbf{項目}              & \textbf{結果}                                                                                                                                                                                                          \\ \hline
            サーバー内処理時間                & ストリーミング処理とチャンク処理の処理時間には大きな違いはなかったが、わずかにストリーミング処理の方が短かった。ページング処理では、ストリーミング処理やチャンク処理より2倍ほどの時間がかかったが、イベント数が75万件を超えたあたりでページング処理の方が処理時間が短くなった。 \\ \hline
            アプリケーションサーバーコンテナのCPU使用率  & ストリーミング処理とチャンク処理ではCPU使用率が終始100\%付近だったが、ページング処理では50\%から90\%の間で変動していた。                                                                                                                                                   \\ \hline
            DBサーバーコンテナのCPU使用率        & どの方式でもCPU使用率に大きな差は確認できなかった。                                                                                                                                                                                                   \\ \hline
            アプリケーションサーバーコンテナのメモリ使用量  & すべての方法でメモリ使用量は同じ傾きで比例しており、顕著な差は確認できなかった。                                                                                                                                                                                   \\ \hline
            DBサーバーコンテナのメモリ使用量        & すべての方法でメモリ使用量は同じ傾きで比例しており、顕著な差は確認できなかった。                                                                                                                                                                                   \\ \hline
            アプリケーションサーバーコンテナのブロック入出力 & すべての方法で75万件のイベント数を超えたあたりからブロック入出力が指数関数的に増加し、方法間での差は確認できなかった。                                                                                                                                                                       \\ \hline
            DBサーバーコンテナのブロック入出力       & どの方法でもイベント数に比例してブロック入出力が増加し、方法間での傾きなどの差は確認できなかった。                                                                                                                                                                                     \\ \hline
            ネットワーク入出力                & すべての方法でネットワーク入出力は同じ傾きで比例しており、量などの差はほとんど確認できなかった。                                                                                                                                                                                   \\ \hline
            最大イベント数                  & チャンク処理では最大800万件、ストリーミング処理では最大900万件まで実行可能だったが、それ以上ではアプリケーションサーバーが落ちた。一方、ページング処理ではどの手法でも2000万件まで実行可能だった。2000万件以上のイベント数については計測していない。                                                    \\ \hline
        \end{tabular}
    \end{table}
\end{document}
