\documentclass[../../../../main]{subfiles}

\begin{document}
    \subsubsection{特定のCPU制限とメモリ制限下の検証結果}\label{subsubsec:result-chunk-only-limit}

    ここでは、Web APIサーバーのアプリケーションサーバーコンテナをCPU制限100\%、メモリ制限1024MBとし、DBサーバーも同様の制限を設定した状態での実験結果を示す。

    \subsubsubsection{イベント数とCPU使用率の関係}\label{subsubsubsec:result-chunk-only-limit-cpu}

    図\ref{fig:chunk-cpu-app_1_1024-db_1_1024}に、イベント数に対するCPU使用率の関係を示す。チャンク処理では、どの程度のまとまりごとにに処理を行うか設定できるが、ここではイベント数を10、100、1000、10000ごとに処理を行った場合の結果を示す。

    \subfile{図:detail-server-time/図:detail-server-time}


\end{document}