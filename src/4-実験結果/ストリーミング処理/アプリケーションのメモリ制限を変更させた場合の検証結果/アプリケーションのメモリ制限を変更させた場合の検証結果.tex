\documentclass[../../../../main]{subfiles}

\begin{document}
    \subsubsection{アプリケーションサーバーのメモリ制限を変更させた場合の検証結果}\label{subsubsec:result-streaming-change-app-memory}

    Web APIサーバーのアプリケーションサーバーコンテナのCPU制限を400\%とし、DBサーバーコンテナのCPU制限を100\%、メモリ制限を1024MBとして、アプリケーションサーバーコンテナのメモリ制限を1024MB、2048MB、4096MB、8192MBと変更させた場合の実験結果を示す。

    \subsubsubsection{サーバー内処理時間}

    図\ref{fig:stream-change-app-memory-limit-server-time-app_4_db_1_1024}に、CPU制限を変化させた際のサーバー内処理時間を示す。

    \subfile{図:app-server-time/図:app-server-time}

    図\ref{fig:stream-change-app-memory-limit-server-time-app_4_db_1_1024}より、メモリ制限が4096MBの場合と8192MBの場合ではどのイベント数でも比例し、1024MBと2048MBの場合では、途中から比例せずに急激に上昇している。2048MBの計測結果に関しては、ブレがあるため正しい考察は得られないが、4096MB以上のメモリ使用量の場合はサーバー内処理時間が変動するようなボトルネックが存在していないと考えられる。

    \subsubsubsection{アプリケーションサーバーのCPU使用率}

    図\ref{fig:stream-change-app-memory-limit-app-cpu-app_4_db_1_1024}に、メモリ制限を変更した際のアプリケーションサーバーコンテナのCPU使用率を、図\ref{fig:stream-change-app-memory-limit-db-cpu-app_4_db_1_1024}にはDBサーバーコンテナのCPU使用率を示す。

    \subfile{図:app-CPU/図:app-CPU}


    \subfile{図:db-CPU/図:db-CPU}

    図\ref{fig:stream-change-app-memory-limit-app-cpu-app_4_db_1_1024}と図\ref{fig:stream-change-app-memory-limit-db-cpu-app_4_db_1_1024}、共にアプリケーションサーバーのメモリ制限を変更したことによる変化は見られなかった。

    \subsubsubsection{アプリケーションサーバーのメモリ使用量}

    図\ref{fig:stream-change-app-memory-limit-app-memory-app_4_db_1_1024}に、メモリ制限を変更させた際のアプリケーションサーバーコンテナのメモリ使用量を、図\ref{fig:stream-change-app-memory-limit-db-memory-app_4_db_1_1024}にはDBサーバーコンテナのメモリ使用量を示す。

    \subfile{図:app-memory/図:app-memory}

    \subfile{図:db-memory/図:db-memory}

    図\ref{fig:stream-change-app-memory-limit-app-memory-app_4_db_1_1024}では、メモリ制限が1024MBと4096MBの場合には比例するように上昇していき、2048MBと8192MBの場合はほぼ一定になっている。

    図\ref{fig:stream-change-app-memory-limit-db-memory-app_4_db_1_1024}でも、メモリ制限が1024MBと4096MBの場合には比例するように上昇していき、2048MBと8192MBの場合はほぼ一定になっている。

    これらのことから2048MBと8192MBの場合と1024MBと4096MBの場合では、計測結果に差があることがわかる。

    \subsubsubsection{ディスクI/O}

    図\ref{fig:stream-change-app-memory-limit-app-disk-in-app_4_db_1_1024}に、メモリ制限を変更させた際のアプリケーションサーバーコンテナのディスク書き込み量を、図\ref{fig:stream-change-app-memory-limit-app-disk-out-app_4_db_1_1024}にはディスク読み込み量を示す。

    \subfile{図:app-disk-in/図:app-disk-in}
    \subfile{図:app-disk-out/図:app-disk-out}

    計測結果から、 メモリ制限が1024MBと2048MBの際にはディスク読み書きが発生しているが、4096MBと8192MBの際にはディスク読み書きが全く発生していないことがわかった。しかし、2048MBの場合にはイベント数が少ない状態でもディスク読み書きが発生しており、イベント集計処理以外でのディスクの読み書きが発生している可能性がある。

    図\ref{fig:stream-change-app-memory-limit-db-disk-in-app_4_db_1_1024}と図\ref{fig:stream-change-app-memory-limit-db-disk-out-app_4_db_1_1024}に、メモリ制限を変更させた際のDBサーバーコンテナのディスク書き込み量とディスク読み込み量を示す。

    \subfile{図:db-disk-in/図:db-disk-in}
    \subfile{図:db-disk-out/図:db-disk-out}

    DBサーバーコンテナのディスク書き込みはほぼ発生しておらず、DBサーバーコンテナのディスク読み込みでもメモリ制限が2048MBの場合だけイベント数が少ない状態でもディスク読み込みが発生していることがわかる。しかしメモリ制限の変更による影響は見られなかった。

    \subsubsubsection{ネットワークI/O}

    図\ref{fig:stream-change-app-memory-limit-app-net-in-app_4_db_1_1024}に、メモリ制限を変更させた際のアプリケーションサーバーコンテナのネットワーク受信量を、図\ref{fig:stream-change-app-memory-limit-app-net-out-app_4_db_1_1024}にはネットワーク送信量を示す。

    \subfile{図:app-net-in/図:app-net-in}
    \subfile{図:app-net-out/図:app-net-out}

    ネットワーク使用量もメモリ制限の変更による影響は見られなかった。

\end{document}