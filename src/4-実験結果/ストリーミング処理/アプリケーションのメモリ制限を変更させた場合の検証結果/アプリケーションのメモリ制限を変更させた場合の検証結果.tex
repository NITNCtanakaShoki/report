\documentclass[../../../../main]{subfiles}

\begin{document}
    \subsubsection{アプリケーションサーバーのメモリ制限を変更した場合の検証結果}\label{subsubsec:result-streaming-change-app-memory}

    本セクションでは, Web APIサーバーのアプリケーションサーバーコンテナのCPU制限を400\%とし, DBサーバーコンテナのCPU制限を100\%, メモリ制限を1024MBと設定した上で, アプリケーションサ

    ーバーコンテナのメモリ制限を1024MB, 2048MB, 4096MB, 8192MBに変更した際の実験結果を示す.

    \subsubsubsection{サーバー内処理時間}

    図\ref{fig:stream-change-app-memory-limit-server-time-app_4_db_1_1024}に, メモリ制限の変更によるサーバー内処理時間の変化を示す.

    \subfile{図:app-server-time/図:app-server-time}

    図\ref{fig:stream-change-app-memory-limit-server-time-app_4_db_1_1024}の分析から, メモリ制限が4096MBと8192MBの場合, サーバー内処理時間はイベント件数に応じて比例して増加しているが, 1024MBと2048MBの場合は, イベント件数が増加すると急激に処理時間が増加している. この結果は, メモリ制限がサーバー内処理時間に影響を与えていることを示している.

    \subsubsubsection{アプリケーションサーバーのCPU使用率}

    図\ref{fig:stream-change-app-memory-limit-app-cpu-app_4_db_1_1024}に, メモリ制限の変更によるアプリケーションサーバーコンテナのCPU使用率を示す.

    \subfile{図:app-CPU/図:app-CPU}

    図\ref{fig:stream-change-app-memory-limit-app-cpu-app_4_db_1_1024}の結果から, メモリ制限の変更によるCPU使用率の顕著な変化は確認されなかった.

    \subsubsubsection{アプリケーションサーバーのメモリ使用量}

    図\ref{fig:stream-change-app-memory-limit-app-memory-app_4_db_1_1024}に, メモリ制限の変更によるアプリケーションサーバーコンテナのメモリ使用量を示す.

    \subfile{図:app-memory/図:app-memory}

    図\ref{fig:stream-change-app-memory-limit-app-memory-app_4_db_1_1024}からは, メモリ制限が4096MBと8192MBの場合, メモリ使用量はイベント件数と比例して増加する一方で, 1024MBと2048MBではメモリ使用量が急激に増加している. これは, 低いメモリ制限下での仮想メモリの使用増加を示唆している.

    図\ref{fig:stream-change-app-memory-limit-db-memory-app_4_db_1_1024}に, DBサーバーコンテナのメモリ使用量を示す.

    \subfile{図:db-memory/図:db-memory}

    DBサーバーコンテナのメモリ使用量も, メモリ制限の変更による顕著な変化は確認されなかった.

    \subsubsubsection{ディスクI/O}

    図\ref{fig:stream-change-app-memory-limit-app-disk-in-app_4_db_1_1024}に, メモリ制限を変更した際のアプリケーションサーバーコンテナのディスク書き込み量を, 図\ref{fig:stream-change-app-memory-limit-app-disk-out-app_4_db_1_1024}にディスク読み込み量を示す.

    \subfile{図:app-disk-in/図:app-disk-in}
    \subfile{図:app-disk-out/図:app-disk-out}

    これらの図から, メモリ制限が低い場合, 特に1024MBと2048MBの際にディスク読み書きが増加しているが, 4096MBと8192MBではディスクI/Oの増加は見られない. これは, メモリ制限が低い場合, 仮想メモリの使用が増加し, ディスクベースのページングが発生していることを示唆している.

    \subsubsubsection{ネットワークI/O}

    図\ref{fig:stream-change-app-memory-limit-app-net-in-app_4_db_1_1024}に, メモリ制限を変更させた際のアプリケーションサーバーコンテナのネットワーク受信量を, 図\ref{fig:stream-change-app-memory-limit-app-net-out-app_4_db_1_1024}にはネットワーク送信量を示す.

    \subfile{図:app-net-in/図:app-net-in}
    \subfile{図:app-net-out/図:app-net-out}

    ネットワーク使用量もメモリ制限の変更による影響は見られなかった.

\end{document}