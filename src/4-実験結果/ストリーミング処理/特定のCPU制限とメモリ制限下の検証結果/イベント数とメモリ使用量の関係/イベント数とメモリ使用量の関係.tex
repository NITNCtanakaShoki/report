\documentclass[../../../../../main]{subfiles}

\begin{document}
    \subsubsubsection{イベント数とメモリ使用量の関係}\label{subsubsec:result-streaming-only-limit-memory}

    この節では, イベント数の増加に伴うメモリ使用量の変化を検証する. 図\ref{fig:stream-memory-app_1_1024-db_1_1024}に実験結果を示す.

    \subfile{図:stream-memory-app_1_1024-db_1_1024/図:stream-memory-app_1_1024-db_1_1024}

    図\ref{fig:stream-memory-app_1_1024-db_1_1024}の分析から, アプリケーションサーバーのメモリ使用量は , イベント数が0から約4万件まではほぼ一定であるが, 4万から約75万件まではイベント数と比例して増加する傾向が見られる. また, イベント数が75万件を超えるとメモリ使用量が一定になり, この時点で1024MBのメモリ制限に達していることが確認できる.

    当初, ストリーミング処理ではORMモデルによるデコード処理がメモリ使用量に大きな影響を与えると予想されていた. しかし, 実際のメモリ使用量はイベント数に比例する傾向を示した. この原因を探るために, アプリケーションサーバ

    ーからDBサーバーへのクエリログを分析した結果, 図\ref{fig:streaming-query}に示すようなクエリパターンが確認された.

    \subfile{図:ストリーミング処理で発行されるクエリ/図:ストリーミング処理で発行されるクエリ}

    ユーザーのイベント数が100万件の際に実行されたクエリの実行時間は約1064.019ミリ秒であり, サーバー内処理時間全体のわずか0.03\%に相当する. このことから, クエリの実行時間がサーバー内処理時間の大部分を占めることはなく, メモリ使用量の増加に大きな影響を与えているとは考えにくい. したがって, サーバー内処理時間の大半は集計処理に費やされており, 集計後のレコードはメモリから迅速に解放されると考えられる. この結果は, ORMのデコード処理において時間とメモリが必要であることを示唆している.

    さらに, CPU使用率とレスポンス時間の分析から推測されるボトルネックの存在は, メモリリソースに関連する問題である可能性が高いと考えられる.

\end{document}