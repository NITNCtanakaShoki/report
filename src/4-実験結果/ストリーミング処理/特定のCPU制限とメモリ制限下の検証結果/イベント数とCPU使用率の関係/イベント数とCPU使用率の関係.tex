\documentclass[../../../../../../main]{subfiles}

\begin{document}
    \subsubsubsection{イベント件数とCPU使用率の関係}\label{subsubsubsec:result-streaming-only-limit-cpu}

    本節では, イベント件数の増加に伴うアプリケーションサーバーとDBサーバーのCPU使用率の変動を分析する. 図\ref{fig:stream-cpu-app_1_1024-db_1_1024}に, イベント件数に応じたCPU使用率の推移を示す.

    \subfile{図:stream-cpu-app_1_1024-db_1_1024/図:stream-cpu-app_1_1024-db_1_1024}

    図\ref{fig:stream-cpu-app_1_1024-db_1_1024}の分析から, アプリケーションサーバーのCPU使用率はほぼ常に100\%であり, システムの処理能力がフルに活用されていることが示されている. 対照的に, DBサーバーのCPU使用率はイベント件数が35万件までは50\%以下で推移しているが, 35万件を超えると約80\%に増加する傾向が見られる. しかしながら, DBサーバーのCPU使用率が100\%に達することはほとんどないため, DBサーバーが処理のボトルネックになっている可能性は低いと考えられる.

    一方で, イベント件数が約75万件を超えた点で, 両サーバーのCPU使用率に減少が観察される. この現象は, システム内に別のボトルネックが存在し, それがCPUの使用効率を低下させていることを示唆している可能性がある.

\end{document}