\documentclass[../../../../../../main]{subfiles}

\begin{document}
    \subsubsubsection{イベント数と処理時間、レスポンス時間の関係}\label{subsubsubsec:result-streaming-only-limit-cpu}

    ここでは、イベント数の増加に対するCPU使用率の推移を示す。図\ref{fig:stream-cpu-app_1_1024-db_1_1024}に、その結果を示す。

    \subfile{図:stream-cpu-app_1_1024-db_1_1024/図:stream-cpu-app_1_1024-db_1_1024}

    図\ref{fig:stream-cpu-app_1_1024-db_1_1024}から、アプリケーションサーバーのCPU使用率が終始100\%であることが確認できる。対照的に、DBサーバーのCPU使用率はイベント数が35万まで50\%以下で推移しているが、35万を超えると80\%前後に増加している。しかし、DBサーバーのCPU使用率が100\%に達することはほとんどないため、DBサーバーがボトルネックである可能性は低いと考えられる。

    さらに、イベント数が約75万を超えると、両サーバーのCPU使用率が減少していることが観察される。これは、新たなボトルネックが発生し、CPUの使用効率が低下したことを示唆している可能性がある。

\end{document}