\documentclass[../../../../../main]{subfiles}

\begin{document}
    \subsubsubsection{イベント件数とネットワークI/Oの関係}\label{subsubsec:result-streaming-only-limit-diskio}

    本節では, イベント件数の増加に伴うネットワークI/Oの変化について分析する.

    \subfile{図:stream-netio-app_1_1024-db_1_1024/図:stream-netio-app_1_1024-db_1_1024}

    図\ref{fig:stream-netio-app_1_1024-db_1_1024}の解析によると, アプリケーションサーバーのインバウンド(IN)とDBサーバーのアウトバウンド(OUT), およびアプリケーションサーバーのアウトバウンド(OUT)とDBサーバーのインバウンド(IN)がほぼ一致している. これは, アプリケーションサーバーがDBサーバーからデータを受信し, またDBサーバーへデータを送信していることを意味している.

    アプリケーションサーバーのインバウンドとDBサーバーのアウトバウンドはイベント件数に比例して増加しており, これはDBから送信されるデータ量がイベント件数に応じて増加していることを示している.

    一方で, アプリケーションサーバーからDBサーバーへのアウトバウンドは, イベント件数が25万までは緩やかに増加しているものの, その後は増加率が低下している. しかし, DBサーバーへの送信量が微々たるものであり, DBサーバーからの受信量に比べて小さいため, ボトルネックにはなっていないと推測される.

    本研究ではアプリケーションサーバーとDBサーバーが同一のホスト上に配置されているが, 実際の運用環境では異なるマシン上に配置されることが一般的である. この場合, ネットワーク帯域が1Gbpsから10Gbpsで

    あることが多い. したがって, 本研究で観測された1GBを超えるネットワークI/Oが実際の運用環境においてボトルネックになる可能性があることを示唆している. 特に, イベント件数が増加するにつれてネットワークトラフィックが増加し, 帯域制限によってパフォーマンスが影響を受ける可能性がある.

    この分析結果は, 大量のイベント処理を伴うシステム設計においてネットワークリソースの計画が重要であることを示している. ネットワーク帯域の制約がシステム全体のパフォーマンスに大きな影響を与え得るため, 適切なネットワーク設計とリソース割り当てが必要である.

\end{document}