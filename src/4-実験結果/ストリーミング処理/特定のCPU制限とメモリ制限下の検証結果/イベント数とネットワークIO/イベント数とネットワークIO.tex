\documentclass[../../../../../main]{subfiles}

\begin{document}
    \subsubsubsection{イベント数とネットワークI/Oの関係}\label{subsubsec:result-streaming-only-limit-diskio}

    ここでは、イベント数の増加に対するネットワークI/Oの推移を記す。

    \subfile{図:stream-netio-app_1_1024-db_1_1024/図:stream-netio-app_1_1024-db_1_1024}

    図\ref{fig:stream-netio-app_1_1024-db_1_1024}から、アプリケーションサーバーのINとDBサーバーのOUT、アプリケーションサーバーのOUTとDBサーバーのINがそれぞれほぼ一致していることがわかる。アプリケーションサーバーのINとDBサーバーのOUTはわかりやすく比例しており、イベント数に応じてDBから送られてくるデータが比例して大きくなることがわかる。

    一方で、アプリケーションサーバーからDBサーバーへの送信は、イベント数が25万あたりまでは緩やかに増加しているが、そこからは増加はしているものの微々たるものである。ここでボトルネックに当たったのかとも思ったが、わずかながらにも少しずつ送信量が増えている上、DBサーバーからアプリケーションサーバーへの送信量に比べれば微々たるものなのでボトルネックにはなっていないかと推測している。

    本研究では外部のネットワークの影響を受けないように、アプリケーションサーバーとDBサーバーは同一のホスト上に存在している。しかし、アプリケーションサーバーとDBサーバーは別マシン上に存在することが多く、その場合は帯域が1Gbpsから10Gbpsになることが多い。そのため、1GBを軽く超えるアプリケーションサーバーとDBサーバーのネットワーク帯域がボトルネックになる可能性が十分に存在すると考えられる。

\end{document}