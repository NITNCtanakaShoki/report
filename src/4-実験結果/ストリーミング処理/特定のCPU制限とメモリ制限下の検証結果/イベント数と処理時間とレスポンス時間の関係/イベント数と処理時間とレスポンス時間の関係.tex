\documentclass[../../../../../main]{subfiles}

\begin{document}
    \subsubsubsection{イベント数と処理時間、レスポンスタイムの関係}\label{subsubsec:result-streaming-only-limit-time}

    イベント数の増加に対する処理時間とレスポンスタイムの関係を図\ref{fig:stream-time-app_1_1024-db_1_1024}に示す。

    \subfile{図:stream-time-app_1_1024-db_1_1024/図:stream-time-app_1_1024-db_1_1024}

    図\ref{fig:stream-time-app_1_1024-db_1_1024}から、サーバー内処理時間とレスポンス時間には顕著な差が見られないことが明らかである。サーバー内処理時間とレスポンス時間の差の平均は、\texttt{0.22745}秒である。また、0秒時点で大幅にサーバー内処理時間とレスポンスタイムに大きなズレがあり、周辺の記録からレスポンスタイムよりサーバー内処理時間の方が正しく適切な指標であると考えられる。

    一方、イベント数が0から75万までの範囲では、レスポンス時間はイベント数におおよそ比例して増加するが、それ以上のイベント数になるとレスポンス時間の増加が急激になる。このことから、70万イベントを超えた辺りで新たなボトルネックが発生している可能性が高いと考えられる。
\end{document}