\documentclass[../../../../../main]{subfiles}

\begin{document}
    \subsubsubsection{イベント数と処理時間、レスポンスタイムの関係}\label{subsubsec:result-streaming-only-limit-time}

    本節では、イベント数の増加に伴う処理時間とレスポンスタイムの関係について調査する。図\ref{fig:stream-time-app_1_1024-db_1_1024}にその関係を示す。

    \subfile{図:stream-time-app_1_1024-db_1_1024/図:stream-time-app_1_1024-db_1_1024}

    図\ref{fig:stream-time-app_1_1024-db_1_1024}の解析から、サーバー内処理時間とレスポンスタイムの間には顕著な差異は見られない。サーバー内処理時間とレスポンスタイムの差の平均は約0.22745秒である。ただし、0秒時点での大幅なズレについては、サーバー内処理時間がレスポンスタイムよりも信頼性の高い指標である可能性が示唆される。

    さらに、イベント数が0から75万までの範囲ではレスポンスタイムがイベント数にほぼ比例して増加することが観察された。しかし、イベント数が75万を超えるとレスポンスタイムの増加率が急激になる。この結果は、イベント数が70万を超えるあたりでシステム内に新たなボトルネックが発生していることを示唆している。

    この分析から、サーバー内処理時間とレスポンスタイムの両方がシステムのパフォーマンスを評価する上で重要であるが、特定のイベント数を超えるとレスポンスタイムがより敏感な指標となり得ることが明らかになった。

\end{document}