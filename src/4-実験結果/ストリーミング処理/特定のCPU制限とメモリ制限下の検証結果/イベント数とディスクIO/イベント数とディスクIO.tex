\documentclass[../../../../../main]{subfiles}

\begin{document}
    \subsubsubsection{イベント数とディスクI/Oの関係}\label{subsubsec:result-streaming-only-limit-diskio}

    イベント数の増加に対するディスクI/Oの推移を記す。

    \subfile{図:stream-disio-app_1_1024-db_1_1024/図:stream-disio-app_1_1024-db_1_1024}

    図\ref{fig:stream-disio-app_1_1024-db_1_1024}から、DBサーバーはほぼOUTのみ、つまり読み込みのみを行っており、書き込みをほとんど行っていないことがわかる。これはイベントソーシングの集計処理の計測であるため予想通りの結果である。またその読み込みはイベント数に対して小さい傾きで比例している。100万のイベント数ではDBサーバーのOUTは約14GBであったが、アプリケーションサーバーのI/Oと比較すると微々たるものであるため、DBサーバーのディスクI/Oがボトルネックになっているとは考えづらい。

    一方、アプリケーションサーバーは、明確に75万イベント未満まではディスク読み書きが0であった。しかし、75万から急激にINとOUTの両方が同じ程度で指数関数的に増加している。75万あたりでメモリが1024MBという上限値に張り付いていたことから、この急激な増加はディスクを仮想メモリとして使用しているためだと推測できる。

\end{document}