\documentclass[../../../../../main]{subfiles}

\begin{document}
    \subsubsubsection{イベント数とディスクI/Oの関係}\label{subsubsubsec:result-streaming-only-limit-diskio}

    本節では, イベント数の増加に伴うディスクI/Oの変化について分析する.

    \subfile{図:stream-disio-app_1_1024-db_1_1024/図:stream-disio-app_1_1024-db_1_1024}

    図\ref{fig:stream-disio-app_1_1024-db_1_1024}の解析から, DBサーバーは主に読み込み操作を行っており, 書き込みはほとんど実施していないことがわかる. この傾向は, イベントソーシングの集計処理が主に読み込みに依存することから予想される結果である. また, DBサーバーの読み込み量はイベント数に比例して増加しているが, 100万イベント数での読み込み量は約14GBに達しており, アプリケーションサーバーのI/Oと比較すると比較的小さな規模である. したがって, DBサーバーのディスクI/Oがボトルネックである可能性は低いと考えられる.

    一方で, アプリケーションサーバーにおけるディスクI/Oは, イベント数が75万件未満ではほぼ0であり, 75万件を超えると急激に読み書きの両方が増加している. これは, メモリ使用量が1024MBの上限に達した後, ディスクスペースを仮想メモリとして使用していることが原因と考えられる. この急激な

    増加は, システムがディスクベースのページングを開始し, 仮想メモリを使用していることを示している. 特に, 75万イベントの閾値を超えた際のディスクの使用率の急増は, メモリリソースが限界に達し, システムが代替のストレージ手段に頼るようになっていることを示唆している.

    この分析から, メモリリソースの制限がシステムパフォーマンスに与える影響を明確に認識することができる. また, ディスクI/Oの動向は, システムのリソース使用状況とボトルネックの特定に役立つ重要な指標であることが分かる.

\end{document}