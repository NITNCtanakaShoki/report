\documentclass[../../../../main]{subfiles}

\begin{document}
    \subsubsection{アプリケーションサーバーのCPU制限を変更させた場合の検証結果}\label{subsubsec:result-streaming-change-app-cpu}

    ここでは、Web APIサーバーのアプリケーションサーバーコンテナのメモリ制限1024MBとし、DBサーバーのCPU制限を100\%、メモリ制限を1024MBとして、アプリケーションサーバーのCPU制限を200\%、300\%、400\%と変更させた場合の実験結果を示す。

    \subsubsubsection{サーバー内処理時間}

    図\ref{fig:stream-change-app-cpu-limit-server-time-app_1024-db_1_1024}に、CPU制限を変化させた際のサーバー内処理時間を示す。

    \subfile{図:app-server-time/図:app-server-time}

    図\ref{fig:stream-change-app-cpu-limit-server-time-app_1024-db_1_1024}より、仮想メモリを使用していないと推測されている75万ほどのイベント数ではほとんどCPU制限が100\%のときと変化がないが、仮想メモリを使い出したと推測される75万のイベント数以降は、サーバー内処理時間が増減していることがわかる。しかし、制限されるCPU量には関係が見られないため、ディスクの状態によって仮想メモリの速度がかわっているのではないかと推測される。

    \subsubsubsection{アプリケーションサーバーのCPU使用率}

    図\ref{fig:stream-change-app-cpu-limit-app-cpu-app_1024-db_1_1024}に、CPU制限を変更した際のアプリケーションサーバーコンテナのCPU使用率を、図\ref{fig:stream-change-app-cpu-limit-db-cpu-app_1024-db_1_1024}にはDBサーバーコンテナのCPU使用率を示す。

    \subfile{図:app-CPU/図:app-CPU}

    \subfile{図:db-CPU/図:db-CPU}

    図\ref{fig:stream-change-app-cpu-limit-app-cpu-app_1024-db_1_1024}と図\ref{fig:stream-change-app-cpu-limit-db-cpu-app_1024-db_1_1024}、共にアプリケーションサーバーのCPU制限を変更したことによる変化は見られなかった。

    \subsubsubsection{アプリケーションサーバーのメモリ使用量}

    図\ref{fig:stream-change-app-cpu-limit-app-memory-app_1024-db_1_1024}に、CPU制限を変更させた際のアプリケーションサーバーコンテナのメモリ使用量を、図\ref{fig:stream-change-app-cpu-limit-db-memory-app_1024-db_1_1024}にはDBサーバーコンテナのメモリ使用量を示す。

    \subfile{図:app-memory/図:app-memory}

    \subfile{図:db-memory/図:db-memory}

    図\ref{fig:stream-change-app-cpu-limit-app-memory-app_1024-db_1_1024}より、比較的にメモリ使用量が増加しているものの、200\%と300\%と400\%で変化がないため、100\%の際のハズレ値の部分が原因であると考えられる。

    また、図\ref{fig:stream-change-app-cpu-limit-db-memory-app_1024-db_1_1024}においては、CPU制限が200\%以上の場合と比較してCPU制限が100\%の際にメモリ使用量が60万付近のイベント数移行急激に増加していることがわかる。しかし、原因を推測するには至らなかった。

    \subsubsubsection{ディスクI/O}

    図\ref{fig:stream-change-app-cpu-limit-app-disk-in-app_1024-db_1_1024}に、CPU制限を変更させた際のアプリケーションサーバーコンテナのディスク書き込み量を、図\ref{fig:stream-change-app-cpu-limit-app-disk-out-app_1024-db_1_1024}にはディスク読み込み量を示す。

    \subfile{図:app-disk-in/図:app-disk-in}
    \subfile{図:app-disk-out/図:app-disk-out}

    図\ref{fig:stream-change-app-cpu-limit-app-disk-in-app_1024-db_1_1024}と図\ref{fig:stream-change-app-cpu-limit-app-disk-out-app_1024-db_1_1024}からディスクの読み書きはCPU制限を変更しても変化がないことがわかる。

    図\ref{fig:stream-change-app-cpu-limit-db-disk-in-app_1024-db_1_1024}と図\ref{fig:stream-change-app-cpu-limit-db-disk-out-app_1024-db_1_1024}に、CPU制限を変更させた際のDBサーバーコンテナのディスク書き込み量とディスク読み込み量を示す。

    \subfile{図:db-disk-in/図:db-disk-in}
    \subfile{図:db-disk-out/図:db-disk-out}

    図\ref{fig:stream-change-app-cpu-limit-db-disk-in-app_1024-db_1_1024}では、CPU制限が100\%がイベント数80万件ほどで上昇しているが、わずか126MBであるため誤差の範囲であると考えられる。

    また図\ref{fig:stream-change-app-cpu-limit-db-disk-out-app_1024-db_1_1024}からはアプリケーションサーバーのCPU制限を変更してもDBサーバーのディスク読み込みの量に変化はないことがわかる。

    \subsubsubsection{ネットワークI/O}

    図\ref{fig:stream-change-app-cpu-limit-app-net-in-app_1024-db_1_1024}に、CPU制限を変更させた際のアプリケーションサーバーコンテナのネットワーク受信量を、図\ref{fig:stream-change-app-cpu-limit-app-net-out-app_1024-db_1_1024}にはネットワーク送信量を示す。

    \subfile{図:app-net-in/図:app-net-in}
    \subfile{図:app-net-out/図:app-net-out}

    図\ref{fig:stream-change-app-cpu-limit-app-net-in-app_1024-db_1_1024}と図\ref{fig:stream-change-app-cpu-limit-app-net-out-app_1024-db_1_1024}から、CPU制限が100\%の場合だけ計測結果にばらつきが見られるが、CPU制限を変更してもネットワークの読み書きに変化はないことがわかる。

\end{document}