\documentclass[../../../main]{subfiles}

\begin{document}
    \begin{table}[H]
        \centering
        \caption{ストリーミング処理での計測結果}
        \label{tab:result-streaming}
        \begin{tabular}{|p{4cm}|p{10cm}|}
            \hline
            \textbf{計測項目}                & \textbf{結果}                                                                                                              \\ \hline
            サーバー内処理時間                    & ユーザーのイベント件数が0から75万件まで増加すると, 処理時間は比例して増加し, 75万件を超えると増加率が大きくなり, 測定値は散らばる傾向が見られた. 75万件の時点での処理時間は約10.388秒であった.                     \\ \hline
            レスポンスタイム                     & サーバー内処理時間とほぼ同様の傾向を示し, 平均差は約0.22745秒で, レスポンスタイムの方がわずかに長かった.                                                                  \\ \hline
            アプリケーションサーバーコンテナのCPU使用率      & イベント件数が0から75万件までの間は, ほとんどの時間でCPU使用率が100\%だった. 75万件を超えると, 使用率は93\%から96\%の範囲に低下した.                                              \\ \hline
            DBサーバーコンテナのCPU使用率            & イベント件数の増加とともにCPU使用率は徐々に増加し, 最大で約90\%に達した. 使用率の増加には明確な規則性(比例や指数関数的な増加)は見られなかった.                                               \\ \hline
            アプリケーションサーバーコンテナのメモリ使用量      & イベント件数が増加するにつれてメモリ使用量は比例して増加し, 75万件を超えると上限の1024MiBに達した.                                                                     \\ \hline
            DBサーバーコンテナのメモリ使用量            & メモリ使用量はイベント件数に比例して増加し, 100万件の時点で432MiBに達した.                                                                                 \\ \hline
            アプリケーションサーバーコンテナのブロック入出力     & 75万件まではブロック入出力が0MBだったが, それを超えると指数関数的に増加し, 100万件の時点で259072MBに達した.                                                            \\ \hline
            DBサーバーコンテナのブロック入出力           & 書き込み量はほぼ0MBで一定だったが, 読み込み量はイベント件数に比例して増加し, 100万件時点で13926.4MBだった.                                                              \\ \hline
            ネットワーク入出力                    & アプリケーションサーバーコンテナとDBサーバーコンテナ間のネットワーク通信量は, ほぼ一致していた. アプリケーションサーバーコンテナの送信量は一定で1730MB, 受信量はイベント件数に比例して増加し, 100万件の時点で44339.2MBだった.  \\ \hline
            アプリケーションサーバーコンテナのCPU制限変更時の影響 & CPU制限の変更による顕著な影響は見られなかった.                                                                                                 \\ \hline
            アプリケーションサーバーコンテナのメモリ制限変更時の影響 & メモリ制限を4096MB以上に設定すると, サーバー内処理時間がイベント件数100万件まで比例して増加し, ブロック入出力が全イベント件数で0になる現象が観察された.                                          \\ \hline
            DBサーバーコンテナのCPU制限変更時の影響       & CPU制限の変更による顕著な影響は見られなかった.                                                                                                 \\ \hline
            DBサーバーコンテナのメモリ制限変更時の影響       & メモリ制限の変更による顕著な影響は見られなかった.                                                                                                 \\ \hline
        \end{tabular}
    \end{table}
\end{document}
