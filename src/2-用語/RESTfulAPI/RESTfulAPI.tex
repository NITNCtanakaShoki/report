\documentclass[../../../main]{subfiles}

\begin{document}
    \subsection{RESTful API}\label{subsec:phraseology-restful-api}

    RESTful APIは, RESTと呼ばれる設計原則に従って設計されたHTTPベースのAPIである. 具体的には, 以下のような特性を備えている. \cite{RealWorldHTTP}

    \begin{itemize}
        \item ウェブサーバーを通じてAPIが提供されている
        \item 例えば\texttt{GET /users/[ユーザーID]/repositories}のように, パスに対してHTTPメソッドを送信することでサービスを得る
        \item APIの成功可否はHTTPステータスコードとしてクライアントに通知される
        \item URLはリソースの場所を示し, サービスの顔として重要である
        \item 必要に応じて, GETパラメータやPOSTのボディなどの追加情報を送信することがある
        \item サーバーからのレスポンスには, JSONやXMLのような構造化されたテキスト, 画像データなどが含まれることが多い
    \end{itemize}

    クライアント視点では, サーバーに対して以下のようなことが期待できる. \cite{RealWorldHTTP}

    \begin{itemize}
        \item URLはリソースの階層を示すパスであり, 主に名詞で構成されている
        \item リソースに対してHTTPメソッドを送信することで, リソースの取得, 更新, 追加などの操作を行う
        \item ステータスコードにより, リクエストが正しく処理されたかどうかを判定できる
        \item GETメソッドは何度実行してもリソースの状態を変更しない
        \item クライアント側では, 各リクエストは独立しており, 管理するべきステートは存在しない
        \item トランザクションは存在しない
    \end{itemize}

    Open APIという定義仕様を用いて, APIの仕様を記述することができる. OpenAPIの使用は, コンピューターによる自動処理(コードジェネレーター, ドキュメントレンダリングなど)を想定している. APIはYAMLまたはJSONの形式で表現することが可能である. \cite{APIデザインパターン}

\end{document}
