\documentclass[../../../main]{subfiles}

\begin{document}
    \subsection{ページング処理}\label{subsec:phraseology-paging_pagination}

    本研究ではページング処理は, Webのブログサイトなどでよく見られるリストを複数ページに分割して表示する処理のことを指す. 一度に多くの項目を表示するとパフォーマンスが悪化するのを防ぐために実装されることが多い.

    ページング処理にはSQLのOFFSET句とLIMIT句を用いて行う方法と, ページの最後の項目を用いて次のページの項目を取得するWHERE句とLIMIT句を用いて行う方法がある. 前者はOFFSETの値が多くなるとINDEXを使用していても後のページになるほどアクセスする必要があるレコードが増えるため, パフォーマンスが悪化する. 後者はOFFSETの値が多くなってもアクセスする必要があるレコードの数は一定であるため, パフォーマンスが悪化することはない.

\end{document}