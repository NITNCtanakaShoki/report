\documentclass[../../../main]{subfiles}

\begin{document}
    \subsection{Docker}\label{subsec:phraseology-docker}

    Dockerとは, アプリケーションの開発, 移動, 実行するためのプラットフォームである. 開発環境や実行環境のポータビリティを向上させるために使われることが多いが, 本研究では1つのマシン上でアプリケーションサーバーやDBサーバーをコンテナとして立ち上げ, それぞれでCPU使用率やメモリ使用量を計測するために使用する. \cite{Docker概要}

    \subsubsection{コンテナのCPU制限}\label{subsubsec:phraseology-docker-cpu-limit}

    本研究では, dockerの機能を用いてコンテナに対してCPU制限を行う. その際には, \texttt{limits}と\texttt{reservations}の両方に設定する. \cite{Docker制限}

    \subsubsection{コンテナのメモリ制限}\label{subsubsec:phraseology-docker-memory-limit}

    本研究では, dockerの機能を用いてコンテナに対してメモリ制限を行う. その際には, \texttt{limits}と\texttt{reservations}の両方に設定する. \cite{Docker制限}

    \subsubsection{docker container statsコマンド}\label{subsubsec:phraseology-docker-container-stats}

    \texttt{docker container stats}コマンドを用いて, 実行中のコンテナからライブ・データ・ストリームを取得することができる. 本実験では, このコマンドを用いてコンテナのCPU使用率やメモリ使用量を計測する.

    このコマンドでは, 0.5秒前後に1度, CPU使用率, メモリ使用量, ブロックI/O, ネットワークI/O, プロセス数などを取得できる. これを随時取得し計測用データベースに保存して, アクセス記録と結合することで実験データを得る. \cite{Docker Stats}


\end{document}