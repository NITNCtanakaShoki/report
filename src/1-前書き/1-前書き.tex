\documentclass[../../main]{subfiles}

\begin{document}
    \chapter{前書き}\label{ch:preface}

    本論文では、「イベントソーシングの適用条件の検証」というテーマについて掘り下げる。ソフトウェアアーキテクチャに対する深い興味を持つ私は、これまでパイプラインアーキテクチャ、レイヤードアーキテクチャなど、様々なアーキテクチャの研究を行ってきた。さらに、関数型プログラミングやオブジェクト指向プログラミングなどのプログラミングパラダイムにも注目している。最近では、データ指向プログラミングという新しい手法にも関心が移っている。

    これらのアーキテクチャやプログラミングパラダイムを試してきたものの、共通の問題点が浮き彫りになった。それは、データの永続化に関わる複雑さである。オブジェクト指向やレイヤードアーキテクチャでは、カプセル化を通じてデータの永続化を隠蔽し、ビジネスロジックに集中できるように設計されている。一方、関数型プログラミングでは、入出力をモナドで表現することで、式で記述しやすくしている。しかし、これらのアプローチでも、データの永続化は依然として複雑な問題であり、ビジネスロジックよりもデータ永続化のロジックが複雑になることは珍しくない。

    この問題に対する解決策として、イベントソーシングに注目を向けた。イベントソーシングは、データの変更をイベントとして扱い、不変なイベントから現在の状態を導出することでデータの永続化を行う。この方法は、イベントの永続化を書き込み時にシンプルなものにすることができる。ただし、読み込み時にはイベントから現在の状態を導出するための処理が必要となり、一定の複雑性は避けられない。そのため、イベントソーシングの適用は、データ変更時の複雑性や保存形式が複雑な場合に特に有効である。

    さらに、サービスの規模が大きくなる現代では、アプリケーションを分割することも多く、CRUDモデルでは書き込み時にデータ処理が複雑化する傾向にある。イベントソーシングでは、イベントの記録のみを行い、処理は読み込み時に実施することで、機能が増えても既存のコードに少ない変更で対応できる。また、後方互換性の維持も容易である。

    実ビジネスにおいて、イベントソーシングの適用は技術的な判断が難しい。イベント数の増加は読み取りに時間がかかり、パフォーマンスの問題が発生する可能性がある。イベントソーシングの適用時の目安を作ることは、適用する際の重要な判断材料となる。本研究では、イベントソーシングのイベント数の増加に伴うCPU使用率、I/O、メモリ使用量の変化を調査し、イベントソーシングの処理方法を改善するための提案を行う。

    インターンやアルバイトを通じて、実ビジネス環境での経験を積み、パフォーマンスに関する実用的な知見を得た。イベントソーシングだけでなく、Kafkaデータベースを用いたイベントストリーミング、Kappaアーキテクチャなども検討したが、実ビジネスではRDBの使用が一般的であり、別のデータベースを用いることは複雑性を増すことが多い。そのため、本研究ではRDBでのイベントソーシングの適用を前提とする。

    本論文を通じて、イベントソーシングの適用条件の検証と、その適用によるデータ永続化の複雑性軽減の可能性について考察する。

    \clearpage
\end{document}