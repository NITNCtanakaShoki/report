\documentclass[../../main]{subfiles}

\begin{document}
    \section{前書き}\label{sec:preface}

    \subsection{目的}\label{subsec:preface-purpose}

    本研究の究極の目的は, イベントソーシングの適用可能条件を検証し, 具体的な適用ガイドラインを策定することである. イベントソーシングはデータの変更をイベントとして扱い, その不変のイベントから現在の状態を導出することにより, データ永続性の複雑さを軽減する可能性を秘めている. しかし, 実際のビジネスにおけるイベントソーシングの適用には技術的な判断が困難であり, 特に読み込みの複雑さやパフォーマンスの問題が顕著になることがある. この研究では, イベントソーシングのパフォーマンス, すなわちイベントの数とメモリ使用量, CPU使用時間の相関, データベース, ネットワーク帯域, I/Oのボトルネックを特定し, イベントソーシングの適用における具体的な判断基準を提供することを目指す.

    \subsection{課題}\label{subsec:preface-problem}

    イベントソーシングの適用に際して直面する主な課題は, データ永続性の複雑さを減らしつつ, 読み込みの複雑さとパフォーマンスの問題をいかに克服するかである. イベントソーシングでは, イベントを全て読み込み, 処理する必要があるが, イベント件数の増加に伴い読み込み時間が長くなる傾向がある. また, キャッシュの更新頻度の決定は難しく, メモリやCPUの使用に関する懸念も存在する. さらに, 実際のビジネス環境ではRDBMSの使用が一般的であり, 他のデータベースの導入はシステムの複雑化を招く可能性があるため, 本研究はRDBMSを使用したイベントソーシングを前提とする. これらの課題を解決し, イベントソーシングの適用において具体的な判断基準を提供することが, 本研究の主な課題である.

    \subsection{研究背景}\label{subsec:preface-motive}

    ソフトウェア開発には様々なアーキテクチャやパラダイムが存在する. これらはソフトウェア開発の複雑性を, 入出力や外部依存といった実装上の複雑さではなく, 解決したい課題独自の複雑性に集中できるようにする目的を持つ. 多くのソフトウェアが直面する難しい問題として, データ永続性の複雑さがある. データの永続化の複雑性は, 競合やデッドロック, 並行更新などの問題を引き起こす.

    これらの問題に対する対策の一つがイベントソーシングである. イベントソーシングは, データの変更や削除を物理的に行うことで競合やデッドロック, 並行更新などの問題を解消できる. しかし, イベントソーシングには大きな問題が存在する. イベントの件数の増加に伴い, 要求されるストレージ量が増加し, 読み込み時には全てのイベントを読み込む必要があるため, 読み込み時間が増加する.

    この問題点から目を背けて実際の永続化手法としてイベントソーシングを適用するのは困難である. 具体的には, どの程度のイベント件数であれば本番環境の永続化手法として適用可能なのか, イベント件数が増加するにつれてどこがボトルネックになっていくのかを推測できなければ適用は難しい. そこで, 本研究ではイベントソーシングの件数がパフォーマンスにどのような影響を与えるのかを検証し, 具体的な適用ガイドラインを提供することを目指す. ストレージに関してはイベント件数から推測が容易であるため, 本研究では対象としない.
    \clearpage
\end{document}