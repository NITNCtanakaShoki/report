\documentclass[../../main]{subfiles}

\begin{document}
    \section{前書き}\label{sec:preface}

    \subsection{目的}\label{subsec:preface-purpose}

    本研究の究極の目的は, イベントソーシングの適用可能条件を検証し, 具体的な適用ガイドラインを策定することである. イベントソーシングはデータの変更をイベントとして扱い, その不変のイベントから現在の状態を導出することにより, データ永続性の複雑さを軽減する可能性を秘めている. しかし, 実際のビジネスにおけるイベントソーシングの適用には技術的な判断が困難であり, 特に読み込みの複雑さやパフォーマンスの問題が顕著になることがある. この研究では, イベントソーシングのパフォーマンス, すなわちイベントの数とメモリ使用量, CPU使用時間の相関, データベース, ネットワーク帯域, I/Oのボトルネックを特定し, イベントソーシングの適用における具体的な判断基準を提供することを目指す.

    \subsection{課題}\label{subsec:preface-problem}

    イベントソーシングの適用に際して直面する主な課題は, データ永続性の複雑さを減らしつつ, 読み込みの複雑さとパフォーマンスの問題をいかに克服するかである. イベントソーシングでは, イベントを全て読み込み, 処理する必要があるが, イベント件数の増加に伴い読み込み時間が長くなる傾向がある. また, キャッシュの更新頻度の決定は難しく, メモリやCPUの使用に関する懸念も存在する. さらに, 実際のビジネス環境ではRDBMSの使用が一般的であり, 他のデータベースの導入はシステムの複雑化を招く可能性があるため, 本研究はRDBMSを使用したイベントソーシングを前提とする. これらの課題を解決し, イベントソーシングの適用において具体的な判断基準を提供することが, 本研究の主な課題である.

    \subsection{動機}\label{subsec:preface-motive}

    私はソフトウェアアーキテクチャに深い関心を抱き, 様々なアーキテクチャやプログラミングパラダイムを試みてきた. パイプラインアーキテクチャ, レイヤードアーキテクチャ, 関数型プログラミング, オブジェクト指向プログラミングなど, 多岐にわたるアプローチを採用してきたが, それらは最終的にはデータ永続性の複雑さに直面する. オブジェクト指向やレイヤードアーキテクチャはデータ永続性を隠蔽し, ビジネスロジックに集中できるように設計されているが, 関数型プログラミングでは入出力をモナドを用いて表現し, 式として記述することが可能である. しかしながら, これらのアプローチでもデータ永続性は依然として複雑な課題である. このような経験を経て, イベントソーシングに関心を持つに至った. イベントソーシングは, データの変更をイベントとして扱い, イベントから現在の状態を導出することで, データ永続性の複雑さを単純化できる可能性がある. また, サービスの規模拡大に伴い, アプリケーションの分割が一般的となり, CRUDモデルでは書き込み時のデータ処理が増加する. イベントソーシングでは, イベントを書き込むだけで読み取り時に処理を行い, 機能が増えても既存コードへの変更を最小限に抑えることができる. これらの理由から, イベントソーシングの適用に非常に前向きである. しかし, 実際のビジネスでイベントソーシングを適用する際の技術的な判断は困難であるため, 本研究を通じてイベントソーシングの適用における具体的なガイドラインを提供し, この分野に貢献したいと考えている.

    \clearpage
\end{document}