\documentclass[../../main]{subfiles}

\begin{document}
    \chapter{前書き}\label{ch:preface}

    \section{目的}\label{sec:preface-purpose}

    本研究の究極の目的は、イベントソーシングの適用可能条件を検証し、具体的な適用ガイドラインを策定することである。イベントソーシングはデータの変更をイベントとして扱い、その不変のイベントから現在の状態を導出することにより、データ永続性の複雑さを軽減する可能性を秘めている。しかし、実際のビジネスにおけるイベントソーシングの適用には技術的な判断が困難であり、特に読み込みの複雑さやパフォーマンスの問題が顕著になることがある。この研究では、イベントソーシングのパフォーマンス、すなわちイベントの数とメモリ使用量、CPU使用時間の相関、データベース、ネットワーク帯域、I/Oのボトルネックを特定し、イベントソーシングの適用における具体的な判断基準を提供することを目指す。

    \section{課題}\label{sec:preface-problem}

    イベントソーシングの適用に際して直面する主な課題は、データ永続性の複雑さを減らしつつ、読み込みの複雑さとパフォーマンスの問題をいかに克服するかである。イベントソーシングでは、イベントを全て読み込み、処理する必要があるが、イベント数の増加に伴い読み込み時間が長くなる傾向がある。また、キャッシュの更新頻度の決定は難しく、メモリやCPUの使用に関する懸念も存在する。さらに、実際のビジネス環境ではRDBの使用が一般的であり、他のデータベースの導入はシステムの複雑化を招く可能性があるため、本研究はRDBを使用したイベントソーシングを前提とする。これらの課題を解決し、イベントソーシングの適用において具体的な判断基準を提供することが、本研究の主な課題である。

    \section{動機}\label{sec:preface-motive}

    私はソフトウェアアーキテクチャに深い関心を抱き、様々なアーキテクチャやプログラミングパラダイムを試みてきた。パイプラインアーキテクチャ、レイヤードアーキテクチャ、関数型プログラミング、オブジェクト指向プログラミングなど、多岐にわたるアプローチを採用してきたが、それらは最終的にはデータ永続性の複雑さに直面する。オブジェクト指向やレイヤードアーキテクチャはデータ永続性を隠蔽し、ビジネスロジックに集中できるように設計されているが、関数型プログラミングでは入出力をモナドを用いて表現し、式として記述することが可能である。しかしながら、これらのアプローチでもデータ永続性は依然として複雑な課題である。このような経験を経て、イベントソーシングに関心を持つに至った。イベントソーシングは、データの変更をイベントとして扱い、イベントから現在の状態を導出することで、データ永続性の複雑さを単純化できる可能性がある。また、サービスの規模拡大に伴い、アプリケーションの分割が一般的となり、CRUDモデルでは書き込み時のデータ処理が増加する。イベントソーシングでは、イベントを書き込むだけで読み取り時に処理を行い、機能が増えても既存コードへの変更を最小限に抑えることができる。これらの理由から、イベントソーシングの適用に非常に前向きである。しかし、実際のビジネスでイベントソーシングを適用する際の技術的な判断は困難であるため、本研究を通じてイベントソーシングの適用における具体的なガイドラインを提供し、この分野に貢献したいと考えている。

    \clearpage
\end{document}