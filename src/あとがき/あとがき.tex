\documentclass[../../main]{subfiles}

\begin{document}
    \section{あとがき}\label{sec:afterword}

    本研究では, DBサーバーとアプリケーションサーバーが同一マシン上にあるという条件下で, イベントソーシングがイベント数に対してどの程度の時間とメモリを要するかを検証することができた. この研究は, イベントソーシングの実装におけるパフォーマンスの観点から有用な知見を提供するものである.

    しかし, 実際の本番環境では, DBサーバーとアプリケーションサーバーが別のマシンに配置され, 同時に複数のアクセスが行われるのが一般的である. したがって, 本研究で得られた結果が本番環境でそのまま適用可能であるかについては, さらなる検証が必要である.

    また, メモリ不足によるアプリケーションの落ちる問題に対処するためのアプローチを考案し, その一部は成功を収めた. しかし, ページング手法を用いたイベントソーシングでは, 処理時間が長く, アプリケーションサーバーのCPUを十分に活用できていないことが明らかになった. これは, ページング手法のイベントソーシングにおいて, さらなる改善の余地があることを示している.

    それにもかかわらず, 本研究を通じて, 負荷が増加するにつれてシステムがどのように振る舞うかの傾向を把握することができた. この知見は, 実務におけるパフォーマンス検証の際に有用な参考資料となると考えられる.

    \clearpage
\end{document}
