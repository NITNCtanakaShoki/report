\documentclass[../../main]{subfiles}

\begin{document}
    \section{後書き}\label{sec:afterword}

    本研究では, DBサーバーとアプリケーションサーバーが同一マシン上にあるという条件下で, イベントソーシングがイベント件数に対してどの程度の時間とメモリを要するかを検証することができた. この研究は, イベントソーシングの実装におけるパフォーマンスの観点から有用な知見を提供するものである.

    しかし, 実際の本番環境では, DBサーバーとアプリケーションサーバーが別のマシンに配置され, 同時に複数のアクセスが行われるのが一般的である. したがって, 本研究で得られた結果が本番環境でそのまま適用可能であるかについては, さらなる検証が必要である.

    また, 集計処理の重さが変化するとどの程度の影響があるのかが調べられていないため, 明確に式に当てはまるだけでなく, 実際に計測してみる必要がある. 集計処理の計算時間に対する影響調査も今後の課題である.

    本研究では1リクエストのみに対するパフォーマンスを検証しているが, 同時リクエストに対する検証は行っていない. 同時リクエストに対する検証も今後の課題である.

    さらに, キャッシュ付きのイベントソーシングの検証ができていないことも重要な課題である. キャッシュが影響するソフトウェアの複雑さを考慮して, イベントソーシングを適用するかどうかを実際に判断しなければならない.

    それにもかかわらず, 本研究を通じて, 負荷が増加するにつれてシステムがどのように振る舞うかの傾向を把握することができた. この知見は, 実務におけるパフォーマンス検証の際に有用な参考資料となると考えられる.

    \clearpage
\end{document}
