\documentclass[../../main]{subfiles}

\begin{document}
    \section{考察}\label{sec:consideration}

    \subsection{ストリーミング処理}\label{subsec:consideration-streaming}

    ストリーミング処理では、イベント数が75万イベントを超えるとメモリ使用量が上限に達し、サーバー内処理時間とブロック入出力が急激に増加した。この現象は、メモリ使用量が上限に達し仮想メモリが使用されたためと考えられる。また、イベント数が過剰に多い場合、アプリケーションサーバーが落ちる事象が発生した。これはメモリとスワップ領域の不足によるものと推測される。アプリケーションコンテナのメモリ使用量がイベント数に比例した原因は、ORMによるオブジェクトへのマッピング処理の遅さにあると考えられる。ネットワーク帯域の制限がないため、オブジェクトマッピングの速度がボトルネックとなった可能性がある。ネットワークI/Oはイベント数100万件付近で40GB程度であり、クラウド環境の帯域に比べて制限がなかった。CPU使用率は100\%付近で推移し、十分に活用されていた。イベント数が少なければ、ストリーミング処理が最も効率的であると考えられる。しかし、イベント数が少ない場合は全レコードを一括で取得し集計する方がソフトウェアの複雑性を減らすと考えられる。

    \subsection{チャンク処理}\label{subsec:consideration-chunk}

    チャンク処理では、速度とメモリ使用量においてストリーミング処理と大きな違いは確認できなかった。チャンクの大きさを変更しても差がないことから、ストリーミング処理とチャンク処理の違いはほとんどないと判断される。したがって、この実験の結果からはチャンク処理を特別に選択する必要はないと言える。

    \subsection{ページング処理}\label{subsec:consideration-paging}

    ページング処理では、ストリーミング処理の2倍程度の処理時間がかかったが、イベント数が増加しても処理時間の増加は見られず、大量のイベント数でもメモリやスワップ不足によるアプリケーションサーバーコンテナの落ちる事象は発生しなかった。しかし、CPU使用率は70\%付近で推移し、最大限の活用ができていなかった。3つのページング手法を試したが、CPUを最大限活かす手法は見つからなかった。OFFSETを使用する方法とページ末イベントを利用した方法との処理時間に大きな差はなかったが、OFFSETを使用する方法がコードの簡潔さから望ましいと考えられる。

    \subsection{ガイドライン}\label{subsec:consideration-guideline}

    本研究の結果から、イベントソーシングを適用する際のガイドラインを提案する。イベント数が少ない場合はストリーミング処理が最も効率的であるが、メモリ使用量に注意が必要である。イベント数が多い場合や、メモリが限られている環境では、ページング処理が有効であると考えられる。特に、OFFSETを用いたページング処理はコードの簡潔さから推奨される。チャンク処理はストリーミング処理と比べて明確な利点が見られなかったため、特別な状況下でなければ選択の必要はないと考えられる。

    \clearpage
\end{document}
