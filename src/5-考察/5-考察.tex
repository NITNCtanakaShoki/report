\documentclass[../../main]{subfiles}

\begin{document}
    \section{考察}\label{sec:consideration}

    \subsection{ストリーミング処理}\label{subsec:consideration-streaming}

    - ストリーミング処理では、レコードを受信してデコード出来次第順次集計処理しているのにアプリケーションサーバーコンテナのメモリ使用量がイベント数に比例してしまった。
    - 結果、イベント数が75万イベントを超えたあたりでアプリケーションサーバーコンテナのメモリが上限に張り付き、サーバー内処理時間は急激に増加し、ブロック入出力も指数関数的に増加した。
    - この現象は75万イベントを超えてメモリ上限に張り付いたので、仮想メモリを使用し出したからだと考えられる。
    - また、ストリーミング処理ではイベント数が多すぎるとアプリケーションサーバーが落ちることとなった。これはメモリとスワップ領域が足りなくなったためだと考えられる。
    - アプリケーションコンテナのメモリ使用量がイベント数に比例してしまった理由としては、レコードをORMでオブジェクトにマッピングする処理が遅いのが原因だと考えられる。
    - また今回はDBサーバーとアプリケーションサーバーを同一マシン上で動作させており、ネットワーク帯域に制限をかけられていないためどれだけのレコード量であっても高速にアプリケーションサーバーに転送されてしまう。
    - これによって普段は問題にならないオブジェクトへのマッピング速度が問題になってしまった可能性も考えられる。
    - ネットワークI/Oはイベント数100万件付近で40GBほどになっていたが、クラウドなどの帯域ではプライベートでも10Gbpsから20Gbpsであるため、本実験ではネットワーク帯域を制限した環境での検証ができていないため、その部分で実際の環境との差異が生じている可能性も考えられる。
    - また、ストリーミング処理ではアプリケーションコンテナのCPU使用率が100\%付近で推移していため、CPUを十分に活用できていると考えられる。
    - また、少ないイベント数であればどの手法よりも処理時間が小さいため、ストリーミング処理はイベント数が少ない場合に向いていると考えられる。
    - ネットワーク帯域での制限がない場合であるために確かかはわからないが、メモリが全取得と対して変わらないレベルで使用されるので、ストリーミング処理が適用できるようなイベント数が少ない場合は、すべてのレコードを一括で取得して集計する方がソフトウェアも複雑になりにくくていいと考える。

    \subsection{チャンク処理}\label{subsec:consideration-chunk}

    - チャンク処理では、ストリーミング処理よりもメモリの使用量は増えるが速度は高速になるのではないかと考えていたが、実際には速度はストリーミング処理よりも遅く、メモリの使用量もこの実験ではストリーミング処理と同じ程度であった。
    - チャンクの大きさを変更しても差がないことからもストリーミング処理とチャンク処理では差がほとんどないとわかる。
    - よって、チャンク処理はこの実験からはわざわざ選択する必要がないと言える。

    \subsection{ページング処理}\label{subsec:consideration-paging}

    - ページング処理では、ストリーミング処理よりも2倍程度処理時間がかかっていることがわかる。
    - しかし、イベント数が増加しても処理時間が増加せず、大量のイベント数でもメモリとスワップ不足でアプリケーションサーバーコンテナが落ちなかったことから、ページング処理はイベント数が非常に多くなる可能性があり、メモリを大量に用意できない場合には有効であると考えられる。
    - また、ページング処理では、イベント数が増加してもアプリケーションサーバーコンテナのCPU使用率が70\%付近で推移していたため、CPUを十分に活用できていないと考えられる。
    - このため、3つのページング手法を試したがまだCPUを最大限活かす手法は発見できなかった。
    - ページングでは、OFFSETを使用する方法よりもページ末イベントを利用したページング処理の方が処理時間が短くなったが、差がわずかであ理、OFFSETを使用する方法の方がコードが簡潔であるため、OFFSETを使用する方法を採用することが望ましいと考えられる。

    \subsection{ガイドライン}\label{subsec:consideration-guideline}

    \clearpage
\end{document}