\documentclass[../../main]{subfiles}

\begin{document}
    \section{考察}\label{sec:consideration}

    \subsection{ストリーミング処理}\label{subsec:consideration-streaming}

    ストリーミング処理では, イベント件数が75万イベントを超えるとメモリ使用量が上限に達し, サーバー内処理時間とブロック入出力が急激に増加した. この現象は, メモリ使用量が上限に達し仮想メモリが使用されたためと考えられる. また, イベント件数が過剰に多い場合, アプリケーションサーバーが落ちる事象が発生した. これはメモリとスワップ領域の不足によるものと推測される. アプリケーションコンテナのメモリ使用量がイベント件数に比例した原因は, ORMによるオブジェクトへのマッピング処理の遅さにあると考えられる. ネットワーク帯域の制限がないため, オブジェクトマッピングの速度がボトルネックとなった可能性がある. ネットワークI/Oはイベント件数100万件付近で40GB程度であり, クラウド環境の帯域に比べて制限がなかった. CPU使用率は100\%付近で推移し, 十分に活用されていた. イベント件数が少なければ, ストリーミング処理が最も効率的であると考えられる. しかし, イベント件数が少ない場合は全レコードを一括で取得し集計する方がソフトウェアの複雑性を減らすと考えられる.

    \subsection{チャンク処理}\label{subsec:consideration-chunk}

    チャンク処理では, 速度とメモリ使用量においてストリーミング処理と大きな違いは確認できなかった. チャンクの大きさを変更しても差がないことから, ストリーミング処理とチャンク処理の違いはほとんどないと判断される. したがって, この実験の結果からはチャンク処理を特別に選択する必要はないと言える.

    \subsection{ページング処理}\label{subsec:consideration-paging}

    ページング処理では, ストリーミング処理の2倍程度の処理時間がかかったが, イベント件数が増加しても処理時間の増加は見られず, 大量のイベント件数でもメモリやスワップ不足によるアプリケーションサーバーコンテナの落ちる事象は発生しなかった. しかし, CPU使用率は70\%付近で推移し, 最大限の活用ができていなかった. 3つのページング手法を試したが, CPUを最大限活かす手法は見つからなかった. OFFSETを使用する方法とページ末イベントを利用した方法との処理時間に大きな差はなかったが, OFFSETを使用する方法がコードの簡潔さから望ましいと考えられる.

    \subsection{ガイドライン}\label{subsec:consideration-guideline}

    本研究から得られるガイドラインは, アプリケーションサーバーとDBサーバーが同一マシン上に存在し, 集計処理が重くない場合に限定される. 以下に具体的な指針を示す.

    まず, 最低要件として, 最高処理時間, 最大メモリ使用量, 最大コア数, 1秒間あたりの最大同時処理数, 最大イベント件数を設定する. 次に, 最高処理時間に対して可能な最大イベント件数を算出する. 本研究の条件下では, ストリーミング処理では以下の式が適用される. $ イベント件数[万件] = 最大処理時間[ms] / 138.5 $ この式で計算したイベント件数が最大イベント件数を超える場合は, キャッシュを利用するか, イベントソーシングの適用を諦める必要がある.

    次に, メモリの消費量を計算する. 本研究の条件であれば, ストリーミング処理の場合, $ メモリ使用量[MiB] = 最大イベント件数[万件] \times 13.653 $ となる. この式で算出したメモリ使用量に同時処理数を掛けた値が最大メモリ使用量を超える場合は, キャッシュの利用やページング処理を検討するか, イベントソーシングの適用を諦める必要がある.

    ページングを利用して処理する場合は, 処理時間が2倍程度かかることを見越す必要がある. さらに, イベントソーシングでは処理にCPUをほとんど最大限使用するため, 1秒あたりの最大同時処理数がCPUコア数と処理時間で計算された値を超える場合には, キャッシュの利用を検討するか, イベントソーシングの適用を諦める必要がある.

    \clearpage
\end{document}
