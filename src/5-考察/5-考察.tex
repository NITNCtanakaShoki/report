\documentclass[../../main]{subfiles}

\begin{document}
    \section{考察}\label{sec:consideration}
    この研究では、特にストリーミング処理、チャンク処理、ページング処理の三つのデータ処理方法を用いてイベントソーシングの実用性と効率性を検証した。それぞれの方法におけるメモリ使用量と処理時間のバランスに注目し、イベント数が増加するにつれて生じる問題点とその解決策について詳細に分析を行った。

    まず、ストリーミング処理では、アプリケーションサーバーのCPUを最大限に活用し、処理時間を短縮することが可能であることが確認された。しかし、イベント数が増加すると、メモリ消費量が比例して増大し、メモリ制限を超えた場合には仮想メモリが使用されることで処理時間が急激に増加する問題が発生した。これは、データベースからのレスポンスをデコードや集計する際に大量のメモリを消費するためである。また、データベースとアプリケーションサーバーが同一ホストに存在することによる転送速度の過剰な高速化が、この問題を悪化させている可能性も示唆された。

    次に、チャンク処理については、ストリーミング処理と比較してサーバー内の処理時間が増加することが観察された。この処理方法では、データを複数のレコードの集合に分割して処理するが、ストリーミング処理が抱えるメモリ問題を解決することはできず、むしろアプリケーションの処理負荷が増大するだけであった。

    一方、ページング処理に関しては、メモリ使用量はストリーミング処理と同程度に増加するものの、サーバー内処理時間がイベント数に比例して一定を保つことが確認された。これは、ページング処理がストリーミング処理やチャンク処理で生じるメモリ不足問題をある程度緩和できることを示唆している。ただし、ページング処理の中でも複数の種類が存在し、それぞれの方法におけるサーバー内処理時間の差は小さいものの、コードの保守性や可読性の観点からは最適とは言えない場合もある。

    この研究から導き出されるイベントソーシングの適用パターンのガイドラインを考えると、バッジ処理のように集計時間が長く、イベント数が非常に多いケースでは、メモリ枯渇よりも処理時間が重要な問題になる可能性がある。このような場合、ページング処理はメモリ問題を解決し、効率的なデータ処理を可能にする。特に、ページング処理においては、OFFSETを使用したクエリ実行が、ページの最後のイベントを利用する方法に比べて、実行時間に大きな差がない上、コードの可読性を維持する点で優れている。

    また、イベント数が比較的少ない場合には、全てのイベントを取得して処理する従来の方法が適していると考えられる。これは、イベント数が少ない場合には、メモリ使用量よりもレスポンス時間がより重要な問題になるためである。この研究によれば、イベント数が少なければ、サーバー内の処理時間はイベント数に比例して予測可能であることがわかっている。そのため、実際のイベント数をもとに、許容可能な処理時間内で処理できるイベントの量を予測し、その範囲内でデータ処理を行うことが重要である。

    さらに、この研究では、ストリーミング処理やチャンク処理においてメモリ不足が続くとアプリケーションが終了する(killされる)問題が確認されたが、ページング処理ではこの問題が発生しなかった。これは、ページング処理がメモリ管理においてより効率的であることを示唆しており、特に大量のデータを扱う場合には、ページング処理が適切な選択肢となる。

    総じて、この卒論研究は、イベントソーシングを効果的に適用するための具体的なガイドラインを提供する。異なる処理方法のメモリ使用量と処理時間のトレードオフを理解し、各ケースに最適な処理方法を選択することが、システムの効率性と実用性を最大化する鍵となる。
    \clearpage
\end{document}