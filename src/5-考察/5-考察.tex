\documentclass[../../main]{subfiles}

\begin{document}
    \section{考察}\label{sec:consideration}

    \subsection{ストリーミング処理}\label{subsec:consideration-streaming}

    ストリーミング処理では, イベント件数が75万イベントを超えるとメモリ使用量が上限に達し, サーバー内処理時間とブロック入出力が急激に増加した. この現象は, メモリ使用量が上限に達し仮想メモリが使用されたためと考えられる. また, イベント件数が過剰に多い場合, アプリケーションサーバーが落ちる事象が発生した. これはメモリとスワップ領域の不足によるものと推測される. アプリケーションコンテナのメモリ使用量がイベント件数に比例した原因は, ORMによるオブジェクトへのマッピング処理の遅さにあると考えられる. ネットワーク帯域の制限がないため, オブジェクトマッピングの速度がボトルネックとなった可能性がある. ネットワークI/Oはイベント件数100万件付近で40GB程度であり, クラウド環境の帯域に比べて制限がなかった. CPU使用率は100\%付近で推移し, 十分に活用されていた. イベント件数が少なければ, ストリーミング処理が最も効率的であると考えられる. しかし, イベント件数が少ない場合は全レコードを一括で取得し集計する方がソフトウェアの複雑性を減らすと考えられる.

    \subsection{チャンク処理}\label{subsec:consideration-chunk}

    チャンク処理では, 速度とメモリ使用量においてストリーミング処理と大きな違いは確認できなかった. チャンクの大きさを変更しても差がないことから, ストリーミング処理とチャンク処理の違いはほとんどないと判断される. したがって, この実験の結果からはストリーミング処理よりもコードが複雑になりやすいチャンク処理を特別に選択する必要はないと言える.

    \subsection{ページング処理}\label{subsec:consideration-paging}

    ページング処理では, ストリーミング処理の2倍程度の処理時間がかかったが, イベント件数が増加しても処理時間の増加は見られず, 大量のイベント件数でもメモリやスワップ不足によるアプリケーションサーバーコンテナの落ちる事象は発生しなかった. しかし, CPU使用率は70\%付近で推移した. つまりCPUが100\%に張り付いていいないということは最大限の活用ができていないということになる. 3つのページング手法を試したが, CPUを最大限活かす手法は見つからなかった. OFFSETを使用する方法とページ末イベントを利用した方法との処理時間に大きな差はなかったが, OFFSETを使用する方法がコードの簡潔さから望ましいと考えられる.

    \subsection{ガイドライン}\label{subsec:consideration-guideline}

    本研究から得られるガイドラインは, 特定の条件下でのみ有効である. 具体的には, アプリケーションサーバーとDBサーバーが同一マシン上に存在し, 集計処理が重くない場合に限定される. 以下に具体的な指針を示す.

    \paragraph{定数の算出方法}
    イベント件数に対する処理時間とメモリ使用量の関係を定量的に示すため, 比例関係にあると仮定して最大イベント件数時のデータから定数を算出する. ここで, \(A\) は処理時間に関する定数, \(B\) はメモリ使用量に関する定数を示す. この比例関係に基づき, 以下の式を定義する.
    \[ イベント件数[万件] = 最大処理時間[ms] / A \]
    \[ メモリ使用量[MiB] = 最大イベント件数[万件] \times B \]
    本研究環境において, 定数\(A\)と\(B\)はそれぞれ\(A = 138.5\), \(B = 13.653\)として算出された.

    \paragraph{理論的根拠}
    これらの式は, イベント処理のパフォーマンスがイベントの数に比例するという観測結果に基づいている. 定数\(A\)と\(B\)は, 最大イベント件数時の実測値をもとに, イベント件数とそれに対応する処理時間やメモリ使用量の比例関係を表す比例係数として導出された. イベント件数にこれらの定数を掛け合わせることで, 予想される処理時間やメモリ使用量を算出することが可能である.

    \paragraph{ガイドラインの有効範囲}
    これらの定数と式は, 実験環境と完全に同一のハードウェアと処理条件下でのみ有効である. 異なる環境では, 定数\(A\)と\(B\)を再計測し, 再評価する必要があり, 定数値はあくまでこの研究の実験環境における目安としてのみ有効である. 異なるドメインやコンピュータ環境での適用時には, 実際に計測した値を用いて適宜調整が必要であり, イベントソーシングの適用を検討する際には, 環境ごとにパフォーマンス特性を評価し, 最適な処理方法を選択する必要がある.

    ページングを利用して処理する場合は, 処理時間が2倍程度かかることを見越す必要がある. さらに, イベントソーシングでは処理にCPUをほとんど最大限使用するため, 1秒あたりの最大同時処理数がCPUコア数と処理時間で計算された値を超える場合には, キャッシュの利用を検討するか, イベントソーシングの適用を諦める必要がある.

    \clearpage
\end{document}
