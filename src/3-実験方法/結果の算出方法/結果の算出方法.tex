\documentclass[../../../main]{subfiles}

\begin{document}
    \subsection{結果の算出方法}\label{subsec:method-calculation}

    本研究では、イベントソーシングのパフォーマンス評価を行うために、Dockerコンテナの状態を監視する\texttt{docker container stats}コマンドから得られるデータを用いて、結果を算出する。このデータにはCPU使用率やメモリ使用量が含まれているが、Web APIサーバーがどの処理を実行している際に取得されたデータかは明確でない。これは、Web APIサーバーがDBサーバーにアクセスしている際、CPU使用量やメモリ使用率が大幅に変動するためである。

    したがって、収集されたデータの中で、CPU使用率やメモリ使用量の最大値を代表値として扱う。このアプローチは、イベントソーシングの適用条件を探る上で、最小値や平均値よりも最大値を基準にすることが適切であると判断したためである。特に、イベントソーシングの適用においては、システムのピーク時の性能が重要であり、これを基準にすることで、システムの限界を把握しやすくなる。

    また、計測の際には、Web APIサーバーが処理中の状態を取得できない可能性もある。このような場合、取得されたデータは実際のシステムの負荷を反映しない低い値となり、誤解を招く可能性がある。したがって、計測時間が短いほど、そのような誤差が生じる可能性が高くなる。

    以下の表は、各計測項目の集計方法を示している。

    \subfile{表:各計測項目の集計方法/表:各計測項目の集計方法}

    \clearpage

\end{document}
