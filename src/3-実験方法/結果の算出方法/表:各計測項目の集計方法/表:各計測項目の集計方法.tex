\documentclass[../../../../main]{subfiles}

\begin{document}
    \begin{table}[htbp]
        \centering
        \caption{各計測項目の最大値の集計方法}
        \label{tab:max-aggregation-method}
        \begin{tabular}{|l|p{6cm}|l|}
            \hline
            \textbf{計測項目} & \textbf{取得方法}                        & \textbf{集計方法}   \\ \hline
            レスポンスタイム      & Web APIにリクエストを投げてレスポンスを受けるまでの時間を計測する & 取得されたデータの中での最低値 \\ \hline
            サーバー内処理時間     & Web APIのアプリケーションサーバー内で集計にかかる時間を計測する  & 取得されたデータの中での最低値 \\ \hline
            CPU使用率        & docker stats                         & 取得されたデータの中での最大値 \\ \hline
            メモリ使用量        & docker stats                         & 取得されたデータの中での最大値 \\ \hline
            ネットワーク入出力     & docker stats                         & 取得されたデータの中での最大値 \\ \hline
            ブロック入出力       & docker stats                         & 取得されたデータの中での最大値 \\ \hline
        \end{tabular}
    \end{table}
\end{document}