\documentclass[../../../main]{subfiles}

\begin{document}
    \subsection{実験手順}\label{subsec:method-procedure}

    本研究の実験手順は以下の通りである. まず, ユーザー間でポイントを取引し, ユーザーのポイントを確認できる機能を持つWeb APIサーバーを構築する. このサーバーでは, ユーザーの作成・削除, ポイント取引のイベントの作成・保存, 過去の取引イベントから現在のユーザーのポイントを導出する機能が実装される. ポイントの導出方法は複数用意される.

    次に, 実験用コンピュータにDockerをインストールし, Web APIサーバーとデータベースサーバーをCPUとメモリに制限をかけた状態で起動する. この環境設定は, リソース使用量の制御を可能にし, 実験の精度を高める.

    計測ツールの準備も重要なステップである. このツールは, \texttt{docker container stats}コマンドを用いて, コンテナのCPU使用率, メモリ使用量, I/O使用量などを定期的に自動で取得し, 計測用のデータベースに保存する. また, Web APIサーバーに対してイベントの作成と現在のポイントの取得を交互に行う. これにより, Web APIサーバーの負荷と応答時間を計測する.

    最後に, Web APIサーバーに対して, CPUやメモリの制限, データベースのインデックスの有無などの条件を変更し, それぞれの条件下での実験を行う. 収集されたデータは, \texttt{docker container stats}で計測された値とアクセス時の計測値を結合して分析され, イベント件数とCPU使用率, メモリ使用量, I/O使用量, 処理時間の相関関係を調査する.

    これらの手順を通じて, イベントソーシングを適用したWeb APIサーバーの性能とリソース使用の相関を明らかにすることを目指す.


\end{document}