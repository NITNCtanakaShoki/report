\documentclass[../../../main]{subfiles}

\begin{document}
    \subsection{計測ツールの構築}\label{subsec:method-measurement_tool}

    イベントソーシングを適用したWeb APIサーバーの性能を正確に測定するために, 専用の計測ツールを開発した. この計測ツールは, SwiftとPostgreSQLで構築されたCLIツールで, 主に二つの機能を持つ.

    まず, 一つ目の機能は, Web APIサーバーが動作しているDockerコンテナの状態を定期的に監視し, そのデータを計測用のデータベースに保存することである. 具体的には, \texttt{docker container stats}コマンドを使用して, コンテナのCPU使用率, メモリ使用量, ネットワーク入出力, ブロック入出力, プロセス数などの情報を取得し, それらのデータとともに計測時刻を記録する.

    以下は計測用データベースのテーブル構造である.

    \subfile{コード:docker_statsテーブルの定義/コード:docker_statsテーブルの定義}

    二つ目の機能は, Web APIサーバーに対してイベントの作成と現在のポイントの取得を交互に行うことである. この機能は, Web APIサーバーへの連続的なアクセスを実現するために設計されている. レスポンスが返ってくるごとに次のリクエストを送信し, それぞれのリクエストについて計測を行う.

    計測データには, APIサーバーとDBサーバーの使用可能なCPU量, メモリ量, インデックスの有無, イベント数, 計測開始時刻, 集計方法, APIサーバー内での処理時間, レスポンス時間, 取得したポイントなどが含まれる.

    以下は計測データを保存するためのテーブル構造である.

    \subfile{コード:measuresテーブルの定義/コード:measuresテーブルの定義}

    measuresテーブルは, 集計方法に対してORMの機能で特定の値のみを許可するようにしている.

    この計測ツールのを用いて本実験の計測を行う. 計測ツールもWeb APIサーバーと同じコンピュータ上で実行されるため, ローカルでやり取りがすみ, インターネットを経由しないため, 計測結果にネットワークの影響を受けない. また, 計測ツールのPostgreSQLデータベースはDockerで起動され, CLIツールはホストOSで直接実行される.

\end{document}