\documentclass[../../main]{subfiles}

\begin{document}
    \section{要約}\label{sec:phraseology}

    本研究では、イベントソーシングパターンの適用条件を検証する。具体的には、イベント数に対するCPU使用率やメモリ使用量、ネットワークI/O、処理時間などの関係を集計方法ごとに検証し、イベントソーシングが適用可能かを調査するガイドラインを作成する。

    結果として、処理時間が短くなければならない場合、全てのイベントを一度に取得して処理する方法が最も効果的である。この理由は、イベント数が多くなると処理時間が長くなり、メモリ使用量やネットワークI/Oのボトルネックよりも処理時間の問題が優先されるためである。少ないイベント数では、処理時間はイベント数に比例する関係が本研究で検証された。したがって、一度計測して比例関係を前提に許容可能なイベント数を算出すると良い。

    一方、処理時間が長くても問題がない場合は、ページング処理を使用するのが適切である。ページング処理は処理時間を増加させるが、メモリ枯渇を避けるメリットがある。メモリ枯渇はアプリケーションの停止を引き起こすため、特に避けるべき問題である。多数のイベントを扱う場合には、ページング処理が適していると考えられる。ページング方法には、OFFSETを利用したクエリと、ページの最後の項目を利用したクエリの二種類があるが、イベントソーシングで現実的に使用可能なイベント数では大きな差はないため、複雑性が低いOFFSETを利用したクエリを推奨する。

    \clearpage
\end{document}