\documentclass[../../main]{subfiles}

\begin{document}
    \section{要約}\label{sec:phraseology}

    本研究の目的は, イベントソーシングの適用可能条件を検証し, 具体的な適用ガイドラインを策定することである. イベントソーシングは, データの変更をイベントとして扱い, 不変のイベントから現在の状態を導出するアプローチであり, データ永続性の複雑さを軽減する可能性を秘めている. しかし, 実際のビジネスにおける適用には技術的な判断が困難であり, 読み込みの複雑さやパフォーマンスの問題が生じることがある. この研究では, イベントソーシングのパフォーマンス, 特にイベントの数とメモリ使用量, CPU使用時間の相関を分析し, 具体的な判断基準を提供する.

    主な課題は, データ永続性の複雑さを減らしつつ, 読み込みの複雑さとパフォーマンスの問題をどのように克服するかである. イベントソーシングでは, イベントを全て読み込み, 処理する必要があり, イベント数の増加に伴い読み込み時間が長くなる傾向がある. また, キャッシュの更新頻度の決定は難しく, メモリやCPUの使用に関する懸念も存在する. 本研究では, RDBを使用したイベントソーシングを前提とする.

    実験方法については, イベントソーシングにおけるイベント数に対するCPU使用率, メモリ使用率, ディスク使用率の関係を調査するための実験方法を説明する. 実験環境として, AMD Ryzen 5 5600G, 16GB DDR4メモリ, 465.8GB NVMe SSDを搭載したコンピュータを使用し, Dockerを介してアプリケーションサーバーとDBサーバーをコンテナとして立ち上げた. 計測ツールは, docker container statsコマンドを用いて, コンテナのCPU使用率, メモリ使用量を計測し, Web APIサーバーに対してイベントの作成と現在のポイントの取得を交互に行った.

    実験結果からは, ストリーミング処理, チャンク処理, ページング処理の3つの異なるイベント集計方法を評価した. ストリーミング処理とチャンク処理では, サーバー内処理時間に大きな違いは見られなかったが, ページング処理では処理時間が長くなることが明らかになった. また, ページング処理では, CPU使用率が最大限に活用されていないことが示された.

    本研究を通じて, イベントソーシングの適用における具体的な判断基準を提供することができた. しかし, 実際の本番環境では, DBサーバーとアプリケーションサーバーが別のマシンに配置されることが一般的であるため, 本研究で得られた結果がそのまま適用可能であるかはさらなる検証が必要である.

    \clearpage
\end{document}